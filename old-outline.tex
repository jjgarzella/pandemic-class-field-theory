\documentclass[a4paper]{article}

%% Language and font encodings
\usepackage[english]{babel}
\usepackage[utf8x]{inputenc}
\usepackage[T1]{fontenc}

%% Sets page size and margins
\usepackage[a4paper,top=3cm,bottom=2cm,left=3cm,right=3cm,marginparwidth=1.75cm]{geometry}

%% Useful packages
\usepackage{mathptmx,amsmath,colonequals}
\usepackage{amssymb,gensymb,mathrsfs}
\usepackage{amscd,amsthm}
\usepackage{mathtools}
\usepackage{graphicx}
\usepackage[colorinlistoftodos]{todonotes}
\usepackage[colorlinks=true, allcolors=blue]{hyperref}
\usepackage[shortlabels]{enumitem}
\usepackage{extarrows}
\usepackage{tikz,tikz-cd}

\newtheorem{thm}{Theorem}[section]
\newtheorem{lem}[thm]{Lemma}
\newtheorem{cor}[thm]{Corollary}
\newtheorem{prop}[thm]{Proposition}
\newtheorem{conj}[thm]{Conjecture}

%\theoremstyle{definition}
\newtheorem{defn}[thm]{Definition}
\newtheorem{ex}[thm]{Example}
\newtheorem{xca}[thm]{Exercise}

%\theoremstyle{remark}
\newtheorem{rmk}[thm]{Remark}

\newcommand{\arxiv}[1]{\href{http://arxiv.org/abs/#1}{{\tt arXiv:#1}}}

% convenient renaming
\newcommand{\isom}{\cong}
\newcommand{\ins}{\subset}
\newcommand{\dual}{\vee}
\newcommand{\cross}{\times}

% general text operators
\newcommand{\Image}{\operatorname{Im}}
\newcommand{\Coim}{\operatorname{Coim}}
\newcommand{\Ker}{\operatorname{Ker}}
\newcommand{\Coker}{\operatorname{Coker}}
\newcommand{\colim}{\operatorname{colim}}

% arrows
\newcommand{\surj}{\twoheadrightarrow}
\newcommand{\inj}{\xhookrightarrow{}}

% context
\newcommand{\basefield}{K}
\newcommand{\basering}{R}

% algebra
\newcommand{\category}[1]{\textsf{#1}}
\newcommand{\Hom}{\operatorname{Hom}}
\newcommand{\Ext}{\operatorname{Ext}}
\newcommand{\Tor}{\operatorname{Tor}}
% TODO: put a \newcommand for the characteristic

% commutative algebra
%\newcommand{\depth}{\operatorname{depth}}
%\newcommand{\height}{\operatorname{height}}
%\newcommand{\pd}{\operatorname{pd}}
%\newcommand{\injd}{\operatorname{injd}}

% algebraic geometry
\newcommand{\Spec}{\operatorname{Spec}}
\newcommand{\Spa}{\operatorname{Spa}}

% custom ops here

% custom commands here

\title{}
\author{Jack J Garzella}

\begin{document}

\maketitle


Ok, so here's how the proof goes:

\begin{enumerate}[(1)]
	\item Develop the machinery of group cohomology.
		We will gloss over or skip most of this.
	\item Calculate the (cohomological) Brauer group of 
		a local field. 
		There's a big diagram, and our goal is to show that
		one of the morphisms in the diagram is an isomorphism.
		This is the invariant map. 
		We do it first in the unramified case, then
		in the ramified case.
		We will deduce some facts about group cohomology as
		corollaries.
	\item Define the Tate local duality pairing, and prove that it
		is perfect.
	\item Deduce the statements of class field theory from the
		fact that the Tate pairing is perfect.
\end{enumerate}

One thing that could be good to try is to have separate 
arguments for those who understand the derived category and those
who don't.

Current prerequisites
\begin{itemize}
	\item Familiarity with derived functors/derived category
		OR basic knowledge of cohomology and spectral sequences
	\item No motivation for LCFT will be given, this is assumed.
	\item Some familiarity with SES-LES arguemnts
	\item Hilbert's theorem 90
\end{itemize}

(Hopeful) prerequisites for the final version
\begin{itemize}
	\item Some basic familiarity with category theory.
	\item Snake lemma familiarity, and some basic
		cohomology nonsense, though much will be redeveloped
		with the group cohomology stuff.
	\item 
\end{itemize}

Quick reconstruction of talk history

Sriram - functorial properties - 7-23 ish
Wei - invariant isomorphism - 7-30
JJ - cft-8-4-2020 - j-4
Brian?? - proof of local tate duality - 8-11??
Wei?? - LCFT theorems? - ??
Sriram - existence theorem - 8-31??

\section{Group cohomology outline}

General nonsense - homological algebra & derived functors
Concrete nonsense
\begin{itemize}
	\item Specific resolution, relation to Ind representations
	\item explicit descriptions of \(H^{1}\) and \(H^{2}\).
	\item Computation tools:
		dimension shifting for cyclic groups,
		hilbert theorem 90,
		herbrand quotient

\end{itemize}

\end{document}
