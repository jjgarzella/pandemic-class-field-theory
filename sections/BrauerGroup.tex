\documentclass[class=article, crop=false]{standalone}

\begin{document}

\section{The Brauer Group}

Our first main task is to compute the Brauer group of 
a local field. 
For our purposes, the Brauer group \(\Br(K)\) is
\(H^{2}(\Gal(\overline{K} / K),\overline{K}^{\times})\).
%This group has more meaning beyond being a cohomology
%group, but we will not need any of this. 
%The interested reader can consult the Wikipedia article for
%``Brauer group''.
We will show that \(\Br(K) \isom \mathbb{Q} / \mathbb{Z}\), 
functorially. More precisely:  

\begin{theorem} \label{thm:br:calc}
	There is a canonical 
	(up to choice of Frobenius\footnote{
	we can choose an arithmetic or geometric Frobenius}) isomorphism
	\[
		\inv_K:\Br(K)\to\QQ/\ZZ
	\] 
	such that for $L/K$ a finite extension, the diagrams
	\[
	\begin{tikzcd}[row sep=3em, column sep=3em]
		\Br(L)\ar[r,"\res"]\ar[d,swap,"\inv_L", "\simeq"'] 
		& \Br(K)\ar[d,"\inv_K","\simeq"']
		& \Br(L)\ar[d,swap,"\inv_L", "\simeq"'] 
		& \Br(K)\ar[l,"\cor"]\ar[d,"\inv_K","\simeq"']\\
		\QQ/\ZZ\ar[r,"{[L:K]}"] 
		& \QQ/\ZZ
		& \QQ/\ZZ
		& \QQ/\ZZ\ar[l,"id"]
	\end{tikzcd}
	\] 
	commute. 
\end{theorem}

\subsection{The Unramified Case}

We give the construction of 
$\inv_K$ by the ``write down a gigntic diagram'' strategy:

\[
	\begin{tikzcd}[column sep=1em]
		H^{2}(\absgal{K},\overline{K}^{\times})
		\ar[rr,dashed,"\inv_K"] & & 
		\QQ/\ZZ &[-1.5em] \ni &[-1.5em] \phi(\Frob_K)\\
		H^{2}(\gal{K^{ur}}{K},(K^{ur})^{\times})
		\ar[r,"\ord"',"\simeq"]\ar[u,"\inf","\simeq"'] & 
		H^{2}(\gal{K^{ur}}{K},\ZZ)\ar[r,"\delta"',"\simeq"] & 
		H^{1}(\gal{K^{ur}}{K},\QQ/\ZZ)\ar[u,"\ev"',"\simeq"] &[-1.5em] \ni &[-1.5em] \phi\ar[u,mapsto] 
\end{tikzcd}
\] 

Here, \(\inf\) is inflation, \(\ord\) is the induced map on homology
from the valuation on \(K^{\times}\), and
\(\ev\) is evalation a homomorphism (when regarded as homogeneous
cochains).
\(\delta\) is a little bit trickier. It is the connecting 
homomorphism (given by the snake lemma) of a long exact sequence
on cohomology. 
The long exact sequence arises from the short exact sequence
of abelian groups
\[
\begin{tikzcd}
0 \arrow{r}{} & \mathbb{Z} \arrow{r}{} & 
\mathbb{Q} \arrow{r}{} & \mathbb{Q} / \mathbb{Z} \arrow{r}{} & 0
\end{tikzcd}
.\]

Finally, $\Frob_K$ is 
either a geometric or arithmetic Frobenius element for $K^{ur}/K$.

{\color{red} (crossed) hom description of \(H^{1}\) }

\begin{proposition} 
	In the diagram, $\ord$ and $\ev$ are isomorphisms.
\end{proposition}

After this proposition, we still haven't dealt with inflation,
but we have a sort of ``unramified version of'' our map \(\inv_{K}\). 

%TODO: break this proof into
%  the trivial ev remark
%  a star lemma that H^r(--, bbQ) = 0
%  a star lemma that proves the delta thing
%  a non-star lemma that proves the ord thing

\begin{proof}
	The action of $G(K^{ur}/K)$ on $\QQ/\ZZ$ is trivial, so
	\[
	H^{1}(G(K^{ur}/K),\QQ/\ZZ)\simeq\Hom(G(K^{ur}/K),\QQ/\ZZ)
	.\] 
	Since $G(K^{ur}/K)$ is generated by $\Frob_K$, a map from $G(K^{ur}/K)$ is determined entirely by what it does to $\Frob_K$. The map $\left[\frac{a}{b}\right]\mapsto(\Frob_K\mapsto\left[\frac{a}{b}\right])$ is clearly an inverse to $\ev$.\\
	There is a short exact sequence
	\[
		0\to\mathcal{O}_{K^{ur}}^{\times}\to(K^{ur})^{\times}\xrightarrow{\ord}\ZZ\to 0
	.\] 
	The goal is to show that $H^{r}(G(K^{ur}/K),\mathcal{O}_{K^{ur}}^{\times})=0$ for $r=2,3$, so that the induced long exact sequence on cohomology gives the desired $\ord$ isomorphism. Since
	\[
		H^{r}(G(K^{ur}/K),\mathcal{O}_{K^{ur}}^{\times})=\colim_{\stackrel{L/K}{\text{unr., finite}}}H^{r}(G(L/K),\mathcal{O}_L^{\times})
	\] 
	and $G(L/K)$ is cyclic, 
	it suffices to prove that 
	$H^{r}(G(L/K),\mathcal{O}_L^{\times})=0$ 
	for $r=1,2$ by the periodicity of cohomology of cyclic groups
	(Lemma \ref{lem:cohom:cycl}).

	There is a Galois module decomposition $L^{\times}=\mathcal{O}_L^{\times}\times\varpi^{\ZZ}$, and since $L/K$ is unramified, $G(L/K)$ acts trivially on $\pi^{\ZZ}$. So,
	\[
		H^{r}(G(L/K),L^{\times}))\simeq H^{r}(G(L/K),\mathcal{O}_L^{\times})\oplus H^{r}(G(L/K),\ZZ)
	.\] 
	By Hilbert theorem 90, $H^{1}(G(L/K),\mathcal{O}_L^{\times})=0$, so the $r=1$ case is done.\\
	For $r=2$, the cohomology theory of cyclic groups gives
	\[
		H^{2}(G(L/K),\mathcal{O}_L^{\times})\simeq\frac{\ker(\sigma-1)}{\im(\Nm)}
	.\] 
	Since $\Nm_{L/K}:\mathcal{O}_L^{\times}\to\mathcal{O}_K^{\times}$ 
	is surjective 
	{\color{red} (equivalence of categories, reduce to finite field case)},
	this takes care of the 
	$r=2$ case and $\ord$ is subsequently an isomorphism.
\end{proof}

The proof for that \(\delta\) is an isomorphism
uses only cohomological machinery.
We break the proof into two steps.

\begin{lemma} \coproof
	For \(G\) a profinite group, 
	\(H^{r}(G,\mathbb{Q}) = 0\) for all \(r\).
	{\color{red} is this true for r=0??}
\end{lemma}

\begin{proof}
	Since $G(K^{ur}/K)$ is profinite and $\QQ$ is discrete,
	\[
		H^{r}(G(K^{ur}/K),\QQ)\simeq\colim_{\stackrel{H\le G(K^{ur}/K)}{\text{ open normal}}}H^{r}(G(K^{ur}/K)/H,\QQ^{H})
	\] 
	by Corollary \ref{cor:lim:cohom}.
	As $G(K^{ur}/K)/H$ is finite, $mH^{r}(G(K^{ur}/K)/H,\QQ^{H})=0$ where
	\[
	m=[G(K^{ur}/K):H]
	.\] 
	Therefore, $H^{r}(G(K^{ur}/K)/H,\QQ^{H})$ is torsion. 
	The colimit of torsion groups is torsion, 
	so $H^{r}(G(K^{ur}/K),\QQ)$ is also torsion. 
	On the other hand, since $\QQ$ is uniquely divisible, the multiplication by $m$ map $[m]$ induces an isomorphism
	\[
		H^{r}(G(K^{ur}/K),\QQ)\xrightarrow{[m]}H^{r}(G(K^{ur}/K),\QQ)
	\] 
	for any $m\ne 0$. In particular, the multiplication by 
	%$\exp(H^{r}(G(K^{ur}/K),\QQ))$  
	\(m\)
	map is an isomorphism, so $H^{r}(G(K^{ur}/K),\QQ)=0$. 
	{\color{red} replace with \(G\) }
\end{proof}

\begin{proposition} \coproof
	 The map \(\delta\) in the graph is an isomorphism.
\end{proposition}

\begin{proof}
	The long exact sequence on cohomology induced by the short exact sequence
	\[
		0\to\ZZ\to\QQ\to\QQ/\ZZ\to 0
	\] 
	gives us an isomorphism
	\[
		\delta:H^{1}(G(K^{ur}/K),\QQ/\ZZ)\xrightarrow{\simeq}H^{2}(G(K^{ur}/K),\ZZ)
	.\] 
	when we apply the previous lemma.
\end{proof}



At this point, we pause our proof that \(\inv_{K}\) is an isomorphism
to check the commutativity of the diagrams in Theorem \ref{thm:br:calc}.
{\color{red} Once we prove these, the full diagrams will 
follow immediately by functoriality of \(\res\) and \(\cor\)
after we have proven the full \(\inv_{K}\) is an isomorophism.}

\begin{proposition} \label{prop:br:res}
	Let $L/K$ be an extension of finite degree. Then the diagram
	\[
	\begin{tikzcd}
		H^{2}(G(K^{ur}/K),(K^{ur})^{\times})\ar{r}{\res}\ar[d,"\inv_K"',"\simeq"] & H^{2}(G(L^{ur}/L),(L^{ur})^{\times})\ar[d,"\inv_L","\simeq"']\\
		\QQ/\ZZ\ar{r}{[L:K]} & \QQ/\ZZ
	\end{tikzcd}
	\] 
	commutes.
\end{proposition}
\begin{proof}
	The given diagram breaks into three squares:
	\[
	\begin{tikzcd}[column sep=1em]
		&&\phi\ar[r,mapsto] & \phi(\Frob_K)\\[-25pt]
		H^2(G(K^{ur}/K),(K^{ur})^{\times})\ar[r,"\ord_K","\simeq"']\ar[d,"\res"] & H^2(G(K^{ur}/K),\ZZ)\ar[d,"e_{L/K}\cdot\res"] & H^{1}(G(K^{ur}/K),\QQ/\ZZ)\ar[l,"\simeq"]\ar[d,"e_{L/K}\cdot\res"]\ar[r,"\simeq"]& \QQ/\ZZ\ar[d,"f_{L/K}\cdot e_{L/K}"]\\
		H^{2}(G(L^{ur}/L),(L^{ur})^{\times})\ar[r,"\ord_L","\simeq"'] & H^2(G(L^{ur}/L),\ZZ) & H^{1}(G(L^{ur}/L),\QQ/\ZZ)\ar[l,"\simeq"']\ar[r,"\simeq"]\ar[r,"\simeq"] & \QQ/\ZZ\\[-25pt]
		&&\phi\ar[r,mapsto] & \phi(\Frob_K)
	\end{tikzcd}
	\] 
	The result follows since $[L:K]=e_{L/K}\cdot f_{L/K}$.
\end{proof}
Finally,
\begin{corollary} \label{prop:br:cores}
	The diagram
	\[
	\begin{tikzcd}
		H^{2}(G(K^{ur}/K),(K^{ur})^{\times})\ar[d,"\inv_K"',"\simeq"] & H^{2}(G(L^{ur}/L),(L^{ur})^{\times})\ar[l,"\cor"]\ar[d,"\inv_L","\simeq"']\\
		\QQ/\ZZ & \QQ/\ZZ\ar[l,"id"]
	\end{tikzcd}
	\] 
	also commutes.
\end{corollary}
\begin{proof}
	Since $\inv_L\circ\res=[L:K]\circ\inv_K$ and $\inv_L$, $\inv_K$ are isomorphisms,
	\[
		id\circ\inv_L\circ\res=id\circ[L:K]=[L:K]
	.\] 
	As $\res$ is surjective and $\cor\circ\res=[L:K]$, the diagram commutes.
\end{proof}

\subsection{The General Case}

There are two diagrams that we want to keep in mind, both of which were introduced by Sriram. The first one is the theorem we wish to prove about the Brauer group:


\begin{center}
\begin{tikzcd} 
Br(K)   \arrow{r}{res} \arrow{d}[near start]{\sim}[near end]{\text{inv}_K}& Br(L) \arrow{d}[near start]{\sim}[near end]{\text{inv}_L} \\
\mathbb{Q}/\mathbb{Z} \arrow{r}{[L:K]} & \mathbb{Q}/\mathbb{Z}
\end{tikzcd}
\end{center}

{\color{red} Notation issues}

From now on, we abbreviate $G(K^{ur}/K)$ as $A_K$ and $H^2(A_K, (K^{ur})^\times)$ as $H^2(K^{ur})$ to simplify notation. Also, $H^2(G_K, \bar{K}^\times)$ will be $H^2(\bar{K}^\times)$. 
The diagram that we have constructed in order to prove this is the following:


\begin{center}
\begin{tikzcd}
H^2(\bar{K}^\times) \arrow[r, dashed]  & \mathbb{Q}/\mathbb{Z}  \\
H^2(K^{ur}) \arrow{u}[near start]{\sim}[near end]{\text{inf}} \arrow{r}{\text{ord}} & H^2(A_K,\mathbb{Z}) & H^1(A_K, \mathbb{Q}/\mathbb{Z}) \arrow{l}{\delta} \arrow{ul}[near start]{\sim}[near end]{\text{ev}}
\end{tikzcd}
\end{center}

Last time, we showed ord, $\delta$, and ev are isomorphisms. 
Specifically, our calculations were in the unramified case. 
This time, our goal is to show that the inf map is an isomorphism; there are two ways to go about doing that. 
The first is by extending the calculation to ramified extensions and then going about it by using an induction argument on the degree of a general extension $L/K$, starting with the case where $L/K$ is a cyclic extension. 

%The second way is by showing $Br(K^{ur})=0$ via central simple algebras, then applying Hochschild-Serre to this result to get $Br(K)\cong H^2(K^{ur})$ via the inflation map. 
%Stefan may have something to say about this. 

Proceeding by our method:

Let $L/K$ be an arbitrary finite Galois extension of degree $n$. 

We can look at the commutative diagram from last time, which is: 

\begin{center}
\begin{tikzcd} 
H^2(K^{ur})\arrow{r}{res} \arrow{d}[near start]{\sim}[near end]{\text{inv}_K}& H^2(L^{ur}) \arrow{d}[near start]{\sim}[near end]{\text{inv}_L}\\
\mathbb{Q}/\mathbb{Z} \arrow{r}{[L:K]} & \mathbb{Q}/\mathbb{Z}
\end{tikzcd}
\end{center}

Where the bottom map is multiplication by $n$, 
the degree of the extension $[L:K]$. 
This means that the kernel of the top map, 
$res$, is ismorphic to a copy of the $n$ torsion elements in 
$\mathbb{Q}/\mathbb{Z}$, which is $\mathbb{Z}/n\mathbb{Z}$. 
This was stated last time as 
$\frac{1}{n}\mathbb{Z}/\mathbb{Z}$, but these are isomorphic. 

Because Gal$(K^{un}/K)$ is a quotient subgroup of $G_K$, we have the induced inflation map on cohomologies: $\text{inf}_K:H^2(K^{ur})\rightarrow H^2(\bar{K}):=H^2(G_K,\bar{K})$ giving us the following diagram: 

\begin{center}
\begin{tikzcd} 
0\arrow[r] &    \ker(res)\arrow[r, hook]\arrow[d, dashed]&    H^2(K^{ur})\arrow{r}{res} \arrow{d}{\text{inf}_K}&       H^2(L^{ur}) \arrow{d}{\text{inf}_L}\\
0\arrow[r] &        H^2(L/K) \arrow[r, hook] &   H^2(\bar{K}) \arrow[r] & H^2(\bar{L})
\end{tikzcd}
\end{center}


Our end goal is to show that the dotted arrow extending 
the map on the left side is an isomorphism. 
It is already injective by the exactness of the
rows.
This will imply the result that we want, as we will have 
$\forall L/K$ finite, Galois extensions, $H^2(L/K)\cong \ker(\text{res}: H^2(K^{ur})\rightarrow H^2(L^{ur}))$.\\

In particular, taking the direct limit over all finite Galois extensions, 
we will have 
$$H^2(\bar{K}/K)=\lim H^2(L/K)=\lim 
\ker(\text{res}: H^2(K^{ur})\rightarrow H^2(L^{ur})) \cong H^2(K^{ur})$$ 
Here's justification for the last isomorphism. 
We restate two facts from Milne: \\

{\color{red} TODO: go through milne and get the 
references for these, putting them in section 1}
2.1.34: If $H$ is normal in $G$, and $M$ is a $G$ module, and $H^i(H, M)=0 \forall i\in [1, r-1]$ for some $r\in\mathbb{Z}_+$, then
$$H^r(G/H, M^H)\cong \ker(\text{res}:H^r(G,M)\rightarrow H^r(H,M))$$ via the Inf map. \\
2.1.36: if we have a tower of extensions: $K\subset L\subset N$ where $L, N$ are Galois over $K$, then Gal$(N/L)=H$ is a normal subgroup of Gal$(N/K)=G$. By proposition 1.22, $H^1(\text{Gal}(L/K), L^\times)=0$ for any finite Galois extension $L/K$. So we have an exact sequence
$$0\rightarrow H^2(G/H, L^\times)\rightarrow H^2(G, N^\times)\rightarrow H^2(H, N^\times)$$

Here, we take $N$ to be $\bar{K}$ and take the limit of $L$, and the result follows. 

We know that \(\ker(\text{res}) \isom \mathbb{Z} / n\mathbb{Z}\) 
as the restriction map is multiplication by \(n\) from the
unramified case {\color{red} include reference?}.
So $H^2(L/K)$ contains a cyclic group of order $n$.
Now, to prove that the dotted line is an isomorphism, 
it is enough to prove that 
$|H^2(L/K)|\leq n$ as the inflation maps 
inf$_k$ and inf$_L$ are injections {\color{red} why?}

To do so, we construct a $G$-stable open subgroup $V\subset \mathcal{O}_L^\times$ such that $H^r(G, V)=0\forall r\geq 1$. Here $G=Gal(L/K).$ Proceed by using the exponent map: this is a $G$ stable isomorphism in finite index mapping from $\mathcal{O}_L\rightarrow\mathcal{O}_L^\times$ sending open sets to open set. Thus, it is enough to construct a $G$-stable open subgroup in $\mathcal{O}_L$ instead. 

{\color{red} cite milne for normal basis theorem}
By the Normal Basis theorem, there is a basis for $L\cong K[G]$ given by $\{x_\tau \mid x_\tau\in L, \tau\in G\}$. Clearing the denominators in this set, we can take all $x_\tau\in \mathcal{O}_L$. Construct $$V=\sum_\tau \mathcal{O}_Kx_\tau\cong_G\mathcal{O}_K[G]=\text{Ind}_1^G\mathcal{O}_K$$ 
In particular, this construction gives $H^r(G,V)=0\forall r>0$ as $r$ is an induced module. 

Now, we first get the result for when $L/K$ is a cyclic extension: 

{\color{red} this goes in the cohomology chapter}
Recall the Herbrand quotient that Brian defined:

$$h(G,M)=\frac{|H^2(G,M)|}{|H^1(G,M)|}$$

And also the second proposition, that for any finite cyclic group $G$, $h(G,M)=1$. So,

$\mathcal{O}_L^\times/V$ is finite so $1=h(G, V)=h(G, \mathcal{O}_L^\times)=h(L^\times)/h(\mathbb{Z})$

Now, since $\mathbb{Z}$ has the trivial action of $G$, $h(\mathbb{Z})=|G|=n$. So $h(L^\times)=n=\frac{|H^2(G,L^\times)|}{|H^1(G,L^\times)|}=|H^2(G,L^\times)|$ and we are done with the cyclic case. 

Now, to extend it to the general case, we proceed by induction on $[L:K]$. Let $[L:K]$ be an arbitrary finite Galois extension. By solvability of Galois groups, we can find an intermediate $K\subsetneq K'\subsetneq L$ such that $K'/K$ is Galois and cyclic. 
Then, we have the exact sequence 
{\color{red} where does this come from? 
Is it Hochschild-Serre with Hilb90?}
$$0\rightarrow H^2(K'/K)\rightarrow H^2(L/K)\rightarrow H^2(L/K')$$ 
where the group is Gal$(L/K)$. We use the lemma that Brian introduced about the multiplicativity of orders of groups in exact sequences to deduce an upper bound for $|H^2(L/K)|:$


$$|H^2(L/K)|\leq |H^2(L/K')|\cdot|H^2(K'/K)|\leq [L:K']\cdot[K':K]=[L:K]$$

and we are done!

\begin{lemma} \label{lem:br:diags}
	The diagrams in the statement of Theorem \ref{thm:br:calc}
	commute.
\end{lemma}

\begin{proof}
	We already know the unramified case, and we just proved that 
	the inflation maps are isomorphisms.
\end{proof}

\end{document}

