\documentclass[class=article, crop=false]{standalone}
\usepackage{blindtext}
\usepackage{amsmath,amsfonts,amssymb,amsthm,tikz-cd,mathtools}
\usepackage[utf8]{inputenc}
\usetikzlibrary{arrows}
\tikzset{commutative diagrams/.cd,arrow style=tikz,diagrams={>=latex'}}
\usepackage[shortlabels]{enumitem}
\setlength\parindent{0pt}

\newcommand{\ZZ}{\mathbb{Z}}
\newcommand{\CC}{\mathbb{C}}
\newcommand{\QQ}{\mathbb{Q}}
\newcommand{\FF}{\mathbb{F}}
\newcommand{\NN}{\mathbb{N}}
\newcommand{\SL}{\operatorname{SL}}
\newcommand{\Div}{\operatorname{Div}}
\newcommand{\Hom}{\operatorname{Hom}}
\newcommand{\Spec}{\operatorname{Spec}}
\newcommand{\Fun}{\operatorname{Fun}}
\newcommand{\Nat}{\operatorname{Nat}}
\newcommand{\Aut}{\operatorname{Aut}}
\newcommand{\Gal}{\operatorname{Gal}}
\newcommand{\Frob}{\operatorname{Frob}}
\newcommand{\Ind}{\operatorname{Ind}}
\newcommand{\res}{\operatorname{res}}
\newcommand{\cor}{\operatorname{cor}}
\newcommand{\inv}{\operatorname{inv}}
\newcommand{\ord}{\operatorname{ord}}
\newcommand{\ev}{\operatorname{ev}}
\newcommand{\GL}{\operatorname{GL}}
\newcommand{\Br}{\operatorname{Br}}
\newcommand{\C}{\operatorname{C}}
\newcommand{\B}{\mathrm{B}}
\newcommand{\Nm}{\mathrm{Nm}}
\newcommand{\im}{\mathrm{im}}
\newcommand{\frakm}{\mathfrak{m}}
\newcommand{\frakn}{\mathfrak{n}}
\newcommand{\fraka}{\mathfrak{a}}
\newcommand{\frakb}{\mathfrak{b}}
\newcommand{\frakc}{\mathfrak{c}}
\newcommand{\frakp}{\mathfrak{p}}
\newcommand{\frakq}{\mathfrak{q}}
\newcommand{\frakP}{\mathfrak{P}}
\newcommand{\frakO}{\mathfrak{O}}

\theoremstyle{definition}
\newtheorem{definition}{Definition}[section]
\newtheorem{lemma}{Lemma}[section]
\newtheorem{theorem}{Theorem}[section]
\newtheorem{proposition}{Proposition}[section]
\newtheorem{corollary}{Corollary}[section]
\newtheorem{remark}{Remark}[section]
\newtheorem{example}{Example}[section]
\newtheorem{conjecture}{Conjecture}[section]

\begin{document}
\section{Functorial Properties of Cohomology}
Let $M$ be a $G$-module and $M'$ a $G'$-module. Two maps $\alpha:G'\to G$ and $\beta:M\to M'$ are \textbf{compatible} if $\beta(\alpha(g)m)=g(\beta(m))$. If $(\alpha,\beta)$ is a compatible pair, then $(\alpha,\beta)$ defines a homomorphism of complexes
\begin{align*}
	C^{r}(G,M)&\to C^{r}(G',M')\\
	\phi&\mapsto\beta\circ\phi\circ\alpha^{r}
.\end{align*}
One can check this by chasing the following diagram:
\[
\begin{tikzcd}
	\cdots\ar{r}{d^{r-1}} & C^{r}(G,M)\ar{r}{d^{r}}\ar[d] & C^{r+1}(G,M)\ar{r}{d^{r+1}}\ar[d] & \cdots\\
	\cdots\ar{r}{d^{r-1}} & C^{r}(G',M')\ar{r}{d^{r}} & C^{r+1}(G',M')\ar{r}{d^{r+1}} & \cdots\\
\end{tikzcd}
.\]
The homomorphism of complexes induces a homomorphism on cohomology
\[
	H^{r}(G,M)\to H^{r}(G',M')
.\]
\begin{example}
Let $H$ be a subgroup of $G$. For any $H$-module $M$, the map
\begin{align*}
	\Ind^{G}_H(M)&\to M\\
	\phi&\mapsto \phi(1_G)
\end{align*}
is compatible with the inclusion map $H\hookrightarrow G$. The induced homomorphism
\[
	H^{r}(G,\Ind^{G}_H(M))\to H^{r}(H,M)
\] 
is exactly the isomorphism from Shapiro's lemma.
\end{example}
\subsection{Restriction} 
Let $H$ be a subgroup of $G$. Let $\alpha:H\hookrightarrow G$ be the inclusion map and $\beta:M\to M$ be the identity map. Fix $g\in G$, $m\in M$. Since $\beta(\alpha(h)m)=hm=h\beta(m)$, $\alpha$ and $\beta$ are compatible. The induced maps on cohomology are called \textbf{restriction homomorphisms}:
\[
	\res:H^{r}(G,M)\to H^{r}(H,M)
.\] 
Alternatively, $\res$ can be realized as the composition
\[
	H^{r}(G,M)\to H^{r}(G,\Ind^{G}_H(M))\xrightarrow{\overset{\text{Shapiro}}{\simeq}} H^{r}(H,M)
\] 
where the first map is induced by  
\begin{align*}
	M&\to\Ind^{G}_H(M)\\
	m&\mapsto (g\mapsto gm)
\end{align*}
and the second is the isomorphism from Shapiro's lemma.
\subsection{Inflation}
Let $H$ be a normal subgroup of $G$. Let $\alpha:G\to G/H$ be the quotient map, and $\beta:M^{H}\hookrightarrow M$ the inlcusion map. Fix $g\in G$, $m\in M^{H}$. Then 
\begin{align*}
	\beta(\alpha(g)m)&=\beta((gH)m)\\
			 &=\beta(g(Hm))\\
			 &=\beta(gm)\\
			 &=gm\\
			 &=g\beta(m)
.\end{align*}
The induced maps on cohomology in this case are called \textbf{inflation homomorphisms}:
\[
	\inf:H^{r}(G/H,M^{H})\to H^{r}(G,M)
.\] 
\subsection{Corestriction}
Let $H$ be a subgroup of $G$ of finite index $n$. Let $\left\lbrace s_1,\dots, s_n\right\rbrace$ be a set of left coset representatives for $H$ in $G$ so that $G$ decomposes as
\[
	G=\bigcup\limits_{i=1}^{n}s_iH
.\] 
Let $M$ be a $G$-module. Consider the norm map  
\begin{align*}
	\Nm_{G/H}:M^{H}&\to M\\
	m&\mapsto\sum\limits_{i=1}^{n}s_im
.\end{align*}
Suppose $\left\lbrace s_1',\dots, s_n'\right\rbrace$ is another set of coset representatives for $H$ in $G$. For $m\in M^{H}$,
\begin{align*}
	\sum\limits_{i=1}^{n}s_im&=\sum\limits_{i=1}^{n}s_i's_i'^{-1}(s_im)\\
				 &=\sum\limits_{i=1}^{n}s_i'(s_i'^{-1}s_i)m\\
				 &=\sum\limits_{i=1}^{n}s_i'm
,\end{align*}
so $\Nm_{G/H}$ is independent of choice of coset representatives. Furthermore, for $g\in G$, $\left\lbrace gs_1,\dots, gs_n\right\rbrace$ is again a set of coset representatives. Therefore the norm map is in fact a well-defined homomorphism
\begin{align*}
	\Nm_{G/H}:M^{H}\to M^{G}
.\end{align*}
This can be extended ($M^{\bullet}=H^{0}(\bullet,M)$) to \textbf{corestriction homomorphisms}
\[
	\cor:H^{r}(H,M)\to H^{r}(G,M)
\] 
for all $r$ in the following way: for every $G$-module $M$, there is a canonical $G$-module homomorphism
\begin{align*}
	\Ind^{G}_H(M)&\to M\\
	\phi&\mapsto\sum\limits_{i=1}^{n}s_i\phi(s_i^{-1})
.\end{align*}
The induced map on cohomology, when composed with the isomorphism from Shapiro's lemma, gives $\cor$:
\[
	H^{r}(H,M)\xrightarrow{\simeq}H^{r}(G,\Ind^{G}_H(M))\to H^{r}(G,M)
.\] 
\subsection{Some Auxiliary Results}
\begin{proposition}
	For $H$ a subgroup of $G$ of finite index, the composition
	\[
		\cor\circ\res:H^{r}(G,M)\to H^{r}(G,M)
	\] 
	is the multiplication by $[G:H]$ homomorphism.
\end{proposition}
\begin{proof}
	For each $m\in M$, define
	\begin{align*}
		\phi_m:G&\to M\\
		g&\mapsto gm
	.\end{align*}
	Let $S$ be a set of coset representatives for $H$ in $G$, and define
	\begin{align*}
		\psi:\Ind^{G}_H(M)&\to M\\
		\phi&\mapsto\sum\limits_{s\in S}s\phi(s^{-1})
	.\end{align*}
	By the definitions of $\res$ and $\cor$, the composite
	\begin{align*}
		M\xrightarrow{m\mapsto\phi_m}\Ind^{G}_H(M)\xrightarrow{\psi}M
	\end{align*}
	induces exactly the composite $\cor\circ\res$ on cohomology. The result follows since
	\[
		\sum\limits_{s\in S}s\phi_m(s^{-1})=\sum\limits_{s\in S}ss^{-1}m=\sum\limits_{s\in S}m=[G:H]m
	.\] 
\end{proof}
\begin{corollary}\label{cor:mult-triv}
	If $\#G=m$, then $mH^{r}(G,M)=0$ for all $r>0$.
\end{corollary}
\begin{proof}
	By the proposition, the multiplication by $m=[G:\left\lbrace 1\right\rbrace]$ map factors through $H^{r}(\left\lbrace 1\right\rbrace,M)=0$ like so
	\[
		H^{r}(G,M)\xrightarrow{\res}H^{r}(\left\lbrace 1\right\rbrace,M)\xrightarrow{\cor} H^{r}(G,M)
	.\] 
\end{proof}
Let $A$ be an abelian group, $p$ a prime. The subgroup
\[
	A(p)=\left\lbrace g\in A:p^{k}g=0 \text{ for some }k\in\ZZ\right\rbrace
.\] 
\begin{corollary}
	Let $G$ be a finite group, $G_p$ its $p$-Sylow subgroup. For every $G$-module $M$, the restriction map
	\[
		\res:H^{r}(G,M)\to H^{r}(G_p,M)
	\] 
	is injective on the $p$-primary component of $H^{r}(G,M)$.
\end{corollary}
\begin{proof}
	The composite
	\[
		\cor\circ\res:H^{r}(G,M)\to H^{r}(G,M)
	\] 
	is multiplication by $[G:G_p]$, which is not divisible by $p$. It follows then that $\cor\circ\res$ is injective on the $p$-primary component of $H^{r}(G,M)$, so in particular $\res$ is.
\end{proof}
\section{The Brauer Group of a Local Field}
The main result is:
\begin{theorem}
	Let $K/\QQ_p$ be a finite extension. There is a canonical (once we choose a geometric or arithmetic Frobenius) isomorphism
	\[
		\inv_K:\Br(K)\to\QQ/\ZZ
	\] 
	such that for $L/K$ a finite extension, the diagram
	\[
	\begin{tikzcd}[row sep=3em, column sep=3em]
		\Br(L)\ar[r,shift left,"\res"]\ar[d,swap,"\inv_L", "\simeq"'] & \Br(K)\ar[l,shift left,"\cor"]\ar[d,"\inv_K","\simeq"']\\
		\QQ/\ZZ\ar[r,shift left,"id"] & \QQ/\ZZ\ar[l,shift left,"{[L:K]}"]
	\end{tikzcd}
	\] 
	commutes. Here $\Br(K)=H^{2}(G_K,\overline{K}^{\times})$ is the (cohomological) Brauer group.
\end{theorem}
Hilbert Theorem 90 gives $H^{1}(G_L,\overline{L}^{\times})=\left\lbrace1\right\rbrace$, so we can use dimension shifting or the Hochschild-Serre spectral sequence to extend the inflation-restriction exact sequence. This combined with the theorem gives
\[
\begin{tikzcd}
	0\ar[r] & H^{2}(G(L/K),\overline{L}^{\times})\ar[r] & H^{2}(G_K,\overline{K}^{\times})\ar{r}{\res}\ar[d,"\inv_K"',"\simeq"] & H^{2}(G_L,\overline{L}^{\times})\ar[d,"\inv_L","\simeq"']\\
		& & \QQ/\ZZ\ar{r}[swap]{[L:K]} & \QQ/\ZZ
\end{tikzcd}
.\] 
There is therefore an isomorphism $H^{2}(\Gal(L/K),\overline{L}^{\times})\simeq\frac{1}{[L:K]}\ZZ/\ZZ$.\\
\subsection{The Unramified Case}
$\inv_K$ is constructed in the following way:
\[
	\begin{tikzcd}[column sep=1em]
		H^{2}(G_K,\overline{K}^{\times})\ar[rr,dashed,"\inv_K"] & & \QQ/\ZZ &[-1.5em] \ni &[-1.5em] \phi(\Frob_K)\\
		H^{2}(G(K^{ur}/K),(K^{ur})^{\times})\ar[r,"\ord"',"\simeq"]\ar[u,"\inf","\simeq"'] & H^{2}(G(K^{ur}/K,\ZZ)\ar[r,"\delta"',"\simeq"] & H^{1}(G(K^{ur}/K),\QQ/\ZZ)\ar[u,"\ev"',"\simeq"] &[-1.5em] \ni &[-1.5em] \phi\ar[u,mapsto] 
\end{tikzcd}
\] 
where $\Frob_K$ is either a geometric or arithmetic Frobenius element for $K^{ur}/K$.
\begin{proposition}
	In the diagram, $\ord$, $\delta$, and $\ev$ are isomorphisms.
\end{proposition}
\begin{proof}
	The action of $G(K^{ur}/K)$ on $\QQ/\ZZ$ is trivial, so
	\[
	H^{1}(G(K^{ur}/K),\QQ/\ZZ)\simeq\Hom(G(K^{ur}/K),\QQ/\ZZ)
	.\] 
	Since $G(K^{ur}/K)$ is generated by $\Frob_K$, a map from $G(K^{ur}/K)$ is determined entirely by what it does to $\Frob_K$. The map $\left[\frac{a}{b}\right]\mapsto(\Frob_K\mapsto\left[\frac{a}{b}\right])$ is clearly an inverse to $\ev$.\\
	Since $G(K^{ur}/K)$ is profinite and $\QQ$ is discrete,
	\[
		H^{r}(G(K^{ur}/K),\QQ)\simeq\varinjlim_{\stackrel{H\le G(K^{ur}/K)}{\text{ open normal}}}H^{r}(G(K^{ur}/K)/H,\QQ^{H})
	.\] 
	As $G(K^{ur}/K)/H$ is finite, $mH^{r}(G(K^{ur}/K)/H,\QQ^{H})=0$ where
	\[
	m=[G(K^{ur}/K):H]
	.\] 
	Therefore, $H^{r}(G(K^{ur}/K)/H,\QQ^{H})$ is torsion. The direct limit of torsion groups is torsion, so $H^{r}(G(K^{ur}/K),\QQ)$ is also torsion. On the other hand, since $\QQ$ is uniquely divisible, the multiplication by $m$ map $[m]$ induces an isomorphism
	\[
		H^{r}(G(K^{ur}/K),\QQ)\xrightarrow{[m]}H^{r}(G(K^{ur}/K),\QQ)
	\] 
	for any $m\ne 0$. In particular, the multiplication by $\exp(H^{r}(G(K^{ur}/K),\QQ))$  map is an isomorphism, so $H^{r}(G(K^{ur}/K),\QQ)=0$. The long exact sequence on cohomology induced by the short exact sequence
	\[
		0\to\ZZ\to\QQ\to\QQ/\ZZ\to 0
	\] 
	can then be boiled down to an isomorphism
	\[
		\delta:H^{1}(G(K^{ur}/K),\QQ/\ZZ)\xrightarrow{\simeq}H^{2}(G(K^{ur}/K),\ZZ)
	.\] 
	There is a short exact sequence
	\[
		0\to\mathcal{O}_{K^{ur}}^{\times}\to(K^{ur})^{\times}\xrightarrow{\ord}\ZZ\to 0
	.\] 
	The goal is to show that $H^{r}(G(K^{ur}/K),\mathcal{O}_{K^{ur}}^{\times})=0$ for $r=2,3$, so that the induced long exact sequence on cohomology gives the desired $\ord$ isomorphism. Since
	\[
		H^{r}(G(K^{ur}/K),\mathcal{O}_{K^{ur}}^{\times})=\varinjlim_{\stackrel{L/K}{\text{unr., finite}}}H^{r}(G(L/K),\mathcal{O}_L^{\times})
	\] 
	and $G(L/K)$ is cyclic, it suffices to prove that $H^{r}(G(L/K),\mathcal{O}_L^{\times})=0$ for $r=1,2$.\\
	There is a Galois module decomposition $L^{\times}=\mathcal{O}_L^{\times}\times\varpi^{\ZZ}$, and since $L/K$ is unramified, $G(L/K)$ acts trivially on $\pi^{\ZZ}$. So,
	\[
		H^{r}(G(L/K),L^{\times}))\simeq H^{r}(G(L/K),\mathcal{O}_L^{\times})\oplus H^{r}(G(L/K),\ZZ)
	.\] 
	By Hilbert theorem 90, $H^{1}(G(L/K),\mathcal{O}_L^{\times})=0$, so the $r=1$ case is done.\\
	For $r=2$, the cohomology theory of cyclic groups gives
	\[
		H^{2}(G(L/K),\mathcal{O}_L^{\times})\simeq\frac{\ker(\sigma-1)}{\im(\Nm)}
	.\] 
	Since $\Nm_{L/K}:\mathcal{O}_L^{\times}\to\mathcal{O}_K^{\times}$ is surjective (equivalence of categories, reduce to finite field case), this takes care of the $r=2$ case and $\ord$ is subsequently an isomorphism.
\end{proof}
It remains to check compatibility with restriction.
\begin{proposition}
	Let $L/K$ be an extension of finite degree. Then the diagram
	\[
	\begin{tikzcd}
		H^{2}(G(K^{ur}/K),(K^{ur})^{\times})\ar{r}{\res}\ar[d,"\inv_K"',"\simeq"] & H^{2}(G(L^{ur}/L),(L^{ur})^{\times})\ar[d,"\inv_L","\simeq"']\\
		\QQ/\ZZ\ar{r}{[L:K]} & \QQ/\ZZ
	\end{tikzcd}
	\] 
	commutes.
\end{proposition}
\begin{proof}
	The given diagram breaks into three squares:
	\[
	\begin{tikzcd}[column sep=1em]
		&&\phi\ar[r,mapsto] & \phi(\Frob_K)\\[-25pt]
		H^2(G(K^{ur}/K),(K^{ur})^{\times})\ar[r,"\ord_K","\simeq"']\ar[d,"\res"] & H^2(G(K^{ur}/K),\ZZ)\ar[d,"e_{L/K}\cdot\res"] & H^{1}(G(K^{ur}/K),\QQ/\ZZ)\ar[l,"\simeq"]\ar[d,"e_{L/K}\cdot\res"]\ar[r,"\simeq"]& \QQ/\ZZ\ar[d,"f_{L/K}\cdot e_{L/K}"]\\
		H^{2}(G(L^{ur}/L),(L^{ur})^{\times})\ar[r,"\ord_L","\simeq"'] & H^2(G(L^{ur}/L),\ZZ) & H^{1}(G(L^{ur}/L),\QQ/\ZZ)\ar[l,"\simeq"']\ar[r,"\simeq"]\ar[r,"\simeq"] & \QQ/\ZZ\\[-25pt]
		&&\phi\ar[r,mapsto] & \phi(\Frob_K)
	\end{tikzcd}
	\] 
	The result follows since $[L:K]=e_{L/K}\cdot f_{L/K}$.
\end{proof}
Finally,
\begin{corollary}
	The diagram
	\[
	\begin{tikzcd}
		H^{2}(G(K^{ur}/K),(K^{ur})^{\times})\ar[d,"\inv_K"',"\simeq"] & H^{2}(G(L^{ur}/L),(L^{ur})^{\times})\ar[l,"\cor"]\ar[d,"\inv_L","\simeq"']\\
		\QQ/\ZZ & \QQ/\ZZ\ar[l,"id"]
	\end{tikzcd}
	\] 
	also commutes.
\end{corollary}
\begin{proof}
	Since $\inv_L\circ\res=[L:K]\circ\inv_K$ and $\inv_L$, $\inv_K$ are isomorphisms,
	\[
		id\circ\inv_L\circ\res=id\circ[L:K]=[L:K]
	.\] 
	As $\res$ is surjective and $\cor\circ\res=[L:K]$, the diagram commutes.
\end{proof}
\end{document}

