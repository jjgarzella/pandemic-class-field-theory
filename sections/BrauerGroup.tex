\documentclass[class=article, crop=false]{standalone}

\begin{document}

\section{The Brauer Group of a Local Field}
The main result is:
\begin{theorem}
	Let $K/\QQ_p$ be a finite extension. There is a canonical (once we choose a geometric or arithmetic Frobenius) isomorphism
	\[
		\inv_K:\Br(K)\to\QQ/\ZZ
	\] 
	such that for $L/K$ a finite extension, the diagram
	\[
	\begin{tikzcd}[row sep=3em, column sep=3em]
		\Br(L)\ar[r,shift left,"\res"]\ar[d,swap,"\inv_L", "\simeq"'] & \Br(K)\ar[l,shift left,"\cor"]\ar[d,"\inv_K","\simeq"']\\
		\QQ/\ZZ\ar[r,shift left,"id"] & \QQ/\ZZ\ar[l,shift left,"{[L:K]}"]
	\end{tikzcd}
	\] 
	commutes. Here $\Br(K)=H^{2}(G_K,\overline{K}^{\times})$ is the (cohomological) Brauer group.
\end{theorem}
Hilbert Theorem 90 gives $H^{1}(G_L,\overline{L}^{\times})=\left\lbrace1\right\rbrace$, so we can use dimension shifting or the Hochschild-Serre spectral sequence to extend the inflation-restriction exact sequence. This combined with the theorem gives
\[
\begin{tikzcd}
	0\ar[r] & H^{2}(G(L/K),\overline{L}^{\times})\ar[r] & H^{2}(G_K,\overline{K}^{\times})\ar{r}{\res}\ar[d,"\inv_K"',"\simeq"] & H^{2}(G_L,\overline{L}^{\times})\ar[d,"\inv_L","\simeq"']\\
		& & \QQ/\ZZ\ar{r}[swap]{[L:K]} & \QQ/\ZZ
\end{tikzcd}
.\] 
There is therefore an isomorphism $H^{2}(\Gal(L/K),\overline{L}^{\times})\simeq\frac{1}{[L:K]}\ZZ/\ZZ$.\\
\subsection{The Unramified Case}
$\inv_K$ is constructed in the following way:
\[
	\begin{tikzcd}[column sep=1em]
		H^{2}(G_K,\overline{K}^{\times})\ar[rr,dashed,"\inv_K"] & & \QQ/\ZZ &[-1.5em] \ni &[-1.5em] \phi(\Frob_K)\\
		H^{2}(G(K^{ur}/K),(K^{ur})^{\times})\ar[r,"\ord"',"\simeq"]\ar[u,"\inf","\simeq"'] & H^{2}(G(K^{ur}/K,\ZZ)\ar[r,"\delta"',"\simeq"] & H^{1}(G(K^{ur}/K),\QQ/\ZZ)\ar[u,"\ev"',"\simeq"] &[-1.5em] \ni &[-1.5em] \phi\ar[u,mapsto] 
\end{tikzcd}
\] 
where $\Frob_K$ is either a geometric or arithmetic Frobenius element for $K^{ur}/K$.
\begin{proposition}
	In the diagram, $\ord$, $\delta$, and $\ev$ are isomorphisms.
\end{proposition}
\begin{proof}
	The action of $G(K^{ur}/K)$ on $\QQ/\ZZ$ is trivial, so
	\[
	H^{1}(G(K^{ur}/K),\QQ/\ZZ)\simeq\Hom(G(K^{ur}/K),\QQ/\ZZ)
	.\] 
	Since $G(K^{ur}/K)$ is generated by $\Frob_K$, a map from $G(K^{ur}/K)$ is determined entirely by what it does to $\Frob_K$. The map $\left[\frac{a}{b}\right]\mapsto(\Frob_K\mapsto\left[\frac{a}{b}\right])$ is clearly an inverse to $\ev$.\\
	Since $G(K^{ur}/K)$ is profinite and $\QQ$ is discrete,
	\[
		H^{r}(G(K^{ur}/K),\QQ)\simeq\varinjlim_{\stackrel{H\le G(K^{ur}/K)}{\text{ open normal}}}H^{r}(G(K^{ur}/K)/H,\QQ^{H})
	.\] 
	As $G(K^{ur}/K)/H$ is finite, $mH^{r}(G(K^{ur}/K)/H,\QQ^{H})=0$ where
	\[
	m=[G(K^{ur}/K):H]
	.\] 
	Therefore, $H^{r}(G(K^{ur}/K)/H,\QQ^{H})$ is torsion. The direct limit of torsion groups is torsion, so $H^{r}(G(K^{ur}/K),\QQ)$ is also torsion. On the other hand, since $\QQ$ is uniquely divisible, the multiplication by $m$ map $[m]$ induces an isomorphism
	\[
		H^{r}(G(K^{ur}/K),\QQ)\xrightarrow{[m]}H^{r}(G(K^{ur}/K),\QQ)
	\] 
	for any $m\ne 0$. In particular, the multiplication by $\exp(H^{r}(G(K^{ur}/K),\QQ))$  map is an isomorphism, so $H^{r}(G(K^{ur}/K),\QQ)=0$. The long exact sequence on cohomology induced by the short exact sequence
	\[
		0\to\ZZ\to\QQ\to\QQ/\ZZ\to 0
	\] 
	can then be boiled down to an isomorphism
	\[
		\delta:H^{1}(G(K^{ur}/K),\QQ/\ZZ)\xrightarrow{\simeq}H^{2}(G(K^{ur}/K),\ZZ)
	.\] 
	There is a short exact sequence
	\[
		0\to\mathcal{O}_{K^{ur}}^{\times}\to(K^{ur})^{\times}\xrightarrow{\ord}\ZZ\to 0
	.\] 
	The goal is to show that $H^{r}(G(K^{ur}/K),\mathcal{O}_{K^{ur}}^{\times})=0$ for $r=2,3$, so that the induced long exact sequence on cohomology gives the desired $\ord$ isomorphism. Since
	\[
		H^{r}(G(K^{ur}/K),\mathcal{O}_{K^{ur}}^{\times})=\varinjlim_{\stackrel{L/K}{\text{unr., finite}}}H^{r}(G(L/K),\mathcal{O}_L^{\times})
	\] 
	and $G(L/K)$ is cyclic, it suffices to prove that $H^{r}(G(L/K),\mathcal{O}_L^{\times})=0$ for $r=1,2$.\\
	There is a Galois module decomposition $L^{\times}=\mathcal{O}_L^{\times}\times\varpi^{\ZZ}$, and since $L/K$ is unramified, $G(L/K)$ acts trivially on $\pi^{\ZZ}$. So,
	\[
		H^{r}(G(L/K),L^{\times}))\simeq H^{r}(G(L/K),\mathcal{O}_L^{\times})\oplus H^{r}(G(L/K),\ZZ)
	.\] 
	By Hilbert theorem 90, $H^{1}(G(L/K),\mathcal{O}_L^{\times})=0$, so the $r=1$ case is done.\\
	For $r=2$, the cohomology theory of cyclic groups gives
	\[
		H^{2}(G(L/K),\mathcal{O}_L^{\times})\simeq\frac{\ker(\sigma-1)}{\im(\Nm)}
	.\] 
	Since $\Nm_{L/K}:\mathcal{O}_L^{\times}\to\mathcal{O}_K^{\times}$ is surjective (equivalence of categories, reduce to finite field case), this takes care of the $r=2$ case and $\ord$ is subsequently an isomorphism.
\end{proof}
It remains to check compatibility with restriction.
\begin{proposition}
	Let $L/K$ be an extension of finite degree. Then the diagram
	\[
	\begin{tikzcd}
		H^{2}(G(K^{ur}/K),(K^{ur})^{\times})\ar{r}{\res}\ar[d,"\inv_K"',"\simeq"] & H^{2}(G(L^{ur}/L),(L^{ur})^{\times})\ar[d,"\inv_L","\simeq"']\\
		\QQ/\ZZ\ar{r}{[L:K]} & \QQ/\ZZ
	\end{tikzcd}
	\] 
	commutes.
\end{proposition}
\begin{proof}
	The given diagram breaks into three squares:
	\[
	\begin{tikzcd}[column sep=1em]
		&&\phi\ar[r,mapsto] & \phi(\Frob_K)\\[-25pt]
		H^2(G(K^{ur}/K),(K^{ur})^{\times})\ar[r,"\ord_K","\simeq"']\ar[d,"\res"] & H^2(G(K^{ur}/K),\ZZ)\ar[d,"e_{L/K}\cdot\res"] & H^{1}(G(K^{ur}/K),\QQ/\ZZ)\ar[l,"\simeq"]\ar[d,"e_{L/K}\cdot\res"]\ar[r,"\simeq"]& \QQ/\ZZ\ar[d,"f_{L/K}\cdot e_{L/K}"]\\
		H^{2}(G(L^{ur}/L),(L^{ur})^{\times})\ar[r,"\ord_L","\simeq"'] & H^2(G(L^{ur}/L),\ZZ) & H^{1}(G(L^{ur}/L),\QQ/\ZZ)\ar[l,"\simeq"']\ar[r,"\simeq"]\ar[r,"\simeq"] & \QQ/\ZZ\\[-25pt]
		&&\phi\ar[r,mapsto] & \phi(\Frob_K)
	\end{tikzcd}
	\] 
	The result follows since $[L:K]=e_{L/K}\cdot f_{L/K}$.
\end{proof}
Finally,
\begin{corollary}
	The diagram
	\[
	\begin{tikzcd}
		H^{2}(G(K^{ur}/K),(K^{ur})^{\times})\ar[d,"\inv_K"',"\simeq"] & H^{2}(G(L^{ur}/L),(L^{ur})^{\times})\ar[l,"\cor"]\ar[d,"\inv_L","\simeq"']\\
		\QQ/\ZZ & \QQ/\ZZ\ar[l,"id"]
	\end{tikzcd}
	\] 
	also commutes.
\end{corollary}
\begin{proof}
	Since $\inv_L\circ\res=[L:K]\circ\inv_K$ and $\inv_L$, $\inv_K$ are isomorphisms,
	\[
		id\circ\inv_L\circ\res=id\circ[L:K]=[L:K]
	.\] 
	As $\res$ is surjective and $\cor\circ\res=[L:K]$, the diagram commutes.
\end{proof}
\end{document}

