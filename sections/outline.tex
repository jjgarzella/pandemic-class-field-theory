
\section{Sketch of the plan}

%Prerequisites
%\begin{itemize}
%	\item Some familiarity with derived categories/functors,
%		some spectral sequences, and group cohomology
%	\item No motivation for LCFT will be given, this is assumed.
%	%\item Some familiarity with SES-LES arguemnts
%	\item Hilbert's theorem 90
%	\item "dimension shifting" and "kummer theory" arguments
%\end{itemize}

In this expository work, we prove the theorems of 
local class field theory. 
As well-known and long-celebrated as class field theory is, 
there are many excellent sources on the subject. 
Among these are Milne (cite) and Kedlaya (cite), both freely
available online. 

Our motivation for writing up class field theory yet again 
is the method
of proof itself. 
Both Kedlaya and Milne use Tate's theorem on group cohomology
as the key step in establishing the result. 
On the other hand, we will establish reciprocity through
the local Tate duality theorem.
The duality theorem is extremely important in modern number
theory, so a teacher or student 
may want to introduce it early for pedagogical reasons.

We have endeavored to make this exposition usable as a 
rough
``drop-in'' replacement for the corresponding proofs 
in Kedlaya (sections 4.1-4.3) and 
Milne (sections III.1-III.3 and III.5).
In particular, we give no motivation for class field 
theory, we will not mention global class feild theory,
and we give few details on group cohomology,
opting instead to state many fundamental lemmas with references.
Given the option between a ``more explicit'' and a
``more cohomologcal'' approach, we will pick the
more cohomological one.
Terse cohomology arguments--including the ever-intimdating
``spectral sequence''--will be explained in the section on
group cohomology.

We always work in characteristic 0.
What about the norm limitation theorem?

The two main components: the proof itself and the group theory
machinery.
We have attempted to separate them as much as possible.
Do we actually do this?

stars for readers who want to prove lemmas on their own?
smoother exposition in some parts, with boring details buried?
try to do details as complete as possible?
choose to do the same spectral sequence argument out twice

%First, we start by calculating the Brauer group of a local field,
%first in the unramified case and then the ramified case.
%Then, we will use this to establish local Tate duality,
%finally deducing Artin Reciprocity and the Existence Theorem
%as corollaries.
%
%Our approach is similar in spirit to Kedlaya (cite) and 
%Milne (cite). 
%While our method is similar in spirit to these sources, our
%aim specifically is to introduce the local tate duality
%theorem.
%
%Our first order of business will be to calculate
%the Brauer group of a local field.
%Once we know the Brauer group, we deduce a few corollaries,
%and then we will define the Tate local duality pairing.
%We will then prove that this pairing is perfect,
%which will allow us to deduce the theorems of
%local class field theory.
%
%This method can be contrasted with the approach of Milne \cite{milne}
%or Kedlaya \cite{kedlaya}. 
%Both start with the same basic
%calculation of the Brauer group, but use an abstract cohomological 
%theorem due to Tate. 
%This ``Abstract Class Field Theory'' approach 
%works well, but the setup is only really good for one thing:
%proving Class Field Theory. 
%Our motivation (and presumably Taylor's original motivation
%for teaching this in an undergrad class) for choosing local tate 
%duality is to introduce 
%another tool
%that the reader may happen across in their furter study of Number 
%Theory.

\subsection{Notation}

Throughout, we assume that \(K\) is a local field
of charactaristic zero. 
%The strategy ought to also work in the function field case

When we write \(\frac{1}{n}\mathbb{Z} / \mathbb{Z}\), we mean
the cyclic group of order \(n\), considered as a subgroup of
\(\mathbb{Q} / \mathbb{Z}\).

