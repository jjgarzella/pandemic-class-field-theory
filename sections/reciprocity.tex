
\section{Local Artin Reciprocity}

We begin this section with a lemma about 
profinite completions

\begin{lemma} \label{lem:double:dual}
	Let \(A\) be an abelian group such that 
	\(A / nA\) is finite for all \(n\). 
	Then the double dual \(A^{\dual\dual}\) is the
	profinite completion of \(A\), i.e.
	\(A^{\dual\dual} \isom \lim_{n} A / nA\).

	In particular, if \(A\) is profinite, 
	\(A^{\dual\dual} \isom A\).
\end{lemma}

\begin{proof}
	Omitted
	{\color{red} not to bad for \(\mathbb{Z}\), in Patrikis' notes,
    and finite groups are trivial I think.
	For profinite groups, this is a special case of
    Pontryagin duality.
    I don't know how to do the general case}
\end{proof}

Traditionally, the local Artin map is presented as an injective
homomorphism \(K^{\cross} \to \gk^{ab}\) with an assortment of
functoriality properties. 
We will state this version in good time, but on the road there
let us take a chance to admire the following:

\begin{theorem}
	There exists a (suitably canonical) isomorphism
	\[
	r_{K} : \hat{K^{\cross}} \to \gk^{ab}
	\] 
\end{theorem}

Constructing this isomorphism is the meat and potatoes
of our proof of class field theory. 
The rest will be garnishes and side dishes proved using
facts about number theory and compatibility properties
of the Tate pairing.

\begin{proof}
	The main strategy is to consider the dual of both
	groups and proving that they are isomorphic.
	Then, we dualize yet again and conclude the theorem
	by Lemma \ref{lem:double:dual}.

	We have elected to do the first step in a rather 
	explicit (if dry) two-column format:

	\begin{align*}
		\left( \gk^{ab} \right)^{\dual} 
		&\isom \Hom_{\text{cont}} (\gk^{ab},\mathbb{Q} / \mathbb{Z}) \\
		&\isom \Hom_{\text{cont}} (\gk, \mathbb{Q} / \mathbb{Z})
		&& \text{universal property of abelianization} \\
		&\isom \colim_{n} \Hom_{\text{cont}} 
		(\gk, \frac{1}{n}\mathbb{Z} / \mathbb{Z})
		&& (*) \\
		&\isom \colim_{n} H^{1}(\gk, \frac{1}{n}\mathbb{Z} / \mathbb{Z})
		&& \text{Homogeneous cochain definition} \\
		&\isom \colim_{n} 
		\Hom_{} (H^{1}(\gk,\mu_{n}),\mathbb{Q} / \mathbb{Z})
		&& \text{Tate duality theorem} \\
		&\isom \colim_{n} \Hom_{} (K^{\times} / (K^{\times})^{n},
		\mathbb{Q} / \mathbb{Z})
		&& \text{Kummer Theory (Lemma \ref{lem:kummer:theory})} \\
		&\isom \Hom_{\text{cont}} (K^{\times},\mathbb{Q} / \mathbb{Z})
		&& (*) \\
		&\isom (K^{\times})^{\dual}
	\end{align*}
	
	A few remarks about the steps:

	Both of the lines marked \((*)\) are justified by noting
	that any such continuous homomorphism has finite image
	due to the compactness of profinite groups, the discreteness
	of \(\mathbb{Q} / \mathbb{Z}\), and the fact that a
	homomorphism from a compact group to a discrete group has
	finite image.

	The map in the Tate duality step is precisely
	\(\phi \mapsto \left<\phi, \cdot \right>\).
	We could write things in terms of the right argument instead
	of the left, but the result would be the same up to 
	canonical isomorphisms by the symmetry of the Tate pairing.

	Also note: the action for the cohomology is the trivial action.
	That's what allows us to replace it with \(\Hom\) and use
	Kummer theory.
	
	Now we use Lemma \ref{lem:double:dual}
	as \(\gk^{ab}\) is profinite, it is isomorphic to it's double
	dual. Likewise, \(K^{\times}\) is a finite direct sum of
	a profinite group and a finitely generated group, so we have
	that the double dual is \(\hat{K^{\times}}\).
	
	Thus, we take the isomorphism we obtain to be \(r_{K}\)
\end{proof}

Notice that \(\hat{K^{\times}}\) has a concrete descrption:
as  \(\mathcal{O}_{K}^{\times}\) is profinite, we have
\(\hat{K^{\times}} \isom \mathcal{O}_{K}^{\times} \times 
\hat{\langle\varpi\rangle}\), where \(\varpi\) is a uniformizer.

Another remark that may be interesting is the fact that despite all of
the machinery we built out of group cohomology, the only
case of local Tate duality that we used to define the Artin map
had a trivial group action.

Now that we have defined the Artin map on the profinite completion
of \(K^{\times}\), let's state the definition of our more traditional
version:

\begin{theorem}
	There is an injective continuous homomorphism 
	$r_K:K^{\times}\to G_K^{ab}$ with dense image characterized 
	by the commutativity of the diagram
	\[
	\begin{tikzcd}
		& \Hom(K^{\times},\ZZ/n\ZZ)\times K^{\times}/(K^{\times})^{n}\ar[d,shift left=3.5em,"\text{Kummer}","\simeq"']\ar{r}{\text{eval}} & \ZZ/n\ZZ\ar[hookrightarrow]{d}{\cdot\frac{1}{n}}\\
		\Hom(G_k^{ab},\ZZ/n\ZZ)\ar{ur}{r_K^{*}}\ar[equals]{r} & H^{1}(K,\ZZ/n\ZZ)\times H^{1}(K,\mu_n)\ar{r}{\langle\ ,\ \rangle_K} & \QQ/\ZZ 
	\end{tikzcd}
	.\] 
\end{theorem}

\begin{proof}
	To define \(r_{K}\), we simply precompose with the inclusion
	\[
		K^{\cross} \inj \hat{K^{\times}} \xrightarrow{\sim}
		\gk^{ab}
	.\] 
	
	Seeing that \(r_{K}^{*}\) satisfies the diagram
	amounts to seeing that the artin map essential is
	taking \(\phi\) to \(\langle \phi, \cdot \rangle\).

	\(r_{K}\) is injective by construction.
	To see that it is continuous, we only need to note
	that each \(K^{\times} \to K^{\times} / (K^{\times})^{n}\) 
	is continuous, thus making the limit hold in the
	category of topological abelian groups as well.
	%TODO: flesh this out
	The fact that the inclusion has dense image is a general fact
	about the profinite completion.
	%TODO: find/come up with the proof
\end{proof}

Now, we list six properties of the Artin map which are also 
considered a part of class field theory.
We will prove (1) and (2) in this section. 
(3) and (4) are exercises for the reader, and 
(5) and (6) will be proved in the next section.

\begin{theorem}
	The artin map has the following properties:
	\begin{enumerate}[(1)]
		\item For all finite extensions \(L / K\),
			We have the commutative diagram
			\[
			\begin{tikzcd}
			L^{\times} \arrow{r}{r_{L}} \arrow{d}[swap]{N_{L / K}} &
			\absgal{L}^{ab} \arrow{d}{} \\
			K^{\times} \arrow{r}{r_{K}} &
			\gk^{ab}
			\end{tikzcd}
			\]
			The unmarked map comes from the
			universal property of the (profinite) abelianization.
		\item 	
			%TODO: introduce norm groups
			%TODO: show that the quotient of abelian 
			%      profinite galois groups is the abelianization
			%      of the galois group of the galois closure
			If \(L / K\) is finite Galois, then 
			\(K^{\times} / N(L^{\times}) \isom \Gal(L / K)\)
			via the Artin map.
			If \(L / K\) is not Galois, then 
			\(K^{\times} / N(L^{\times})\) is isomorphic
			to the abelianization of \(\Gal(L / K)\),
			where according to standard convention,
			the Galois group of a non-abelian extension
			is defined to be the Galois group of its 
			Galois closure.
		\item 
			We have the ``transfer map'' diagram
			\[
			\begin{tikzcd}
				K^{\times} \arrow{r}{r_{K}} 
				\arrow[hookrightarrow]{d}{} &
			\gk^{ab} \arrow{d}{V} \\
			L^{\times} \arrow{r}{r_{L}} &
			\absgal{L}^{ab}
			\end{tikzcd}
			,\]
			where the transfer map \(V\) comes from 
			the dual of corestriction.
		\item
			If \(a \in K^{\times}\), and 
			\(\chi \in \Hom_{} (\gk,\mathbb{Q} / \mathbb{Z})\),
			by the connecting homomorphism on cohomology
			(using the homogeneous cochains definition)
			we get \(\delta\chi \in H^{2}(\gk, \mathbb{Z})\).
			Then we get 
			\(a \cup \delta \chi \in H^{2}(K,\overline{k}^{\times})\).
			And we have
			\[
				\chi(r_{K}(a) = - \inv_{K}(a \cup \delta\chi)
			.\] 
	\end{enumerate}
\end{theorem}


\begin{proof}
	\begin{enumerate}[(1)]
		\item We need to show that \(r_{K}(N_{L / K}(b) = r_{L}(b)\) 
			for any \(b \in K^{\times}\).
			By Pontryagin duality, we can check this on the
			double dual group.
			Thus, we need to show that
			\(\chi(r_{K}(N_{L / K}(b))) = \chi(r_{L}(b))\) 
			for all \(\chi \in \Hom_{\text{cont}} (\gk^{ab},
			\mathbb{Q} / \mathbb{Z})\).

			By tracing through the characterization diagram
			and the definition of the duality pairing, 
			we have that
			\(\frac{1}{n}\chi(r_{L}(b) = 
			\inv_{L}( \res(x) \cup \delta b)\)
			where \(\delta\) comes from the isomorphism 
			in Kummer Theory (this isomorphishm happens
			to be a connecting map from a long exact sequene)
			and the restriction is across the inclusion 
			\(\absgal{L}^{ab} \inj \gk^{ab}\). 

			By applying the corestriction diagram 
			from the Brauer group
			calculation \ref{thm:br:calc}, 
			this becomes 
			\(\inv_{K}(\cor(\res(\chi) \cup \delta b))\), 
			which by the functoriality properties
			of corestriction, becomes
			\(\inv_{K} (\chi \cap \cor(\delta b)) \). 

			Now, becuase of the obvious commutativity of the
			diagram 
			\[
			\begin{tikzcd}
				L^{\times} / (L^{\times})^{n} \arrow{r}{\delta} 
				\arrow{d}[swap]{N_{L / K}} &
				H^{1}(L, \mu_{n}) \arrow{d}{\cor} \\
				K^{\times} / (K^{\times})^{n} \arrow{r}{\delta} &
				H^{1}(K, \mu_{n})
			\end{tikzcd}
			\]
			our expression becomes 
			\(\inv_{K}(\chi \cup \delta(N_{L / K}(b)))
			= \frac{1}{n} \chi(r_{K}(N_{L / K}(b)))\) 

		\item 
			By the constuction of \(r_{K}\), part (1), 
			and taking cokernels, we have a diagram
			\[
			\begin{tikzcd}
			L^{\times} \arrow{r}{N_{L / K}} \arrow{d}{r_{L}} &
			K^{\times} \arrow{r}{} \arrow{d}{r_{K}} &
			K^{\times} / N_{L / K} (L^{\times}) \arrow{d}{r_{L / K}} \\
			\absgal{L}^{ab} \arrow{r}{} &
			\gk^{ab} \arrow{r}{} &
			\Gal(L / K)
			\end{tikzcd}
			\]
			which defines \(r_{L / K}\). 

			Now, we show that \(r_{L / K}\) is injective and 
			surjective by a diagram chase.

			\textbf{Surjective:}
			Pick \(g \in \Gal(L / K)\). 
			Let \(g^{\prime} \in \gk^{ab}\) 
			be an element which maps to \(g\). 
			As the Artin map has dense image, 
			find a sequence \((a_{i})\) in \(K^{\times}\)
			whose images converge to \(g\) in \(\gk^{ab}\).
			By compactness of \(K^{\times} / N_{L / K}(L^{\times}\),
			it contains a point which the images
			of the \(a_{i}\) converge to, call it \(a\). 
			By commutativity, \(a\) must map to \(g\). 

			\textbf{Injective:} 
			If \(a \in K^{\times}\) maps to \(1\),
			then \(r_{K}(a)\) (which must thusly be
			in \(\absgal{L}^{ab}\)) is
			the limit of images \((b_{i})\) 
			in \(L^{\times}\), under the map
			\(r_{L}\). 
			Thus, \(r_{K}(a) = \lim r_{K}(N_{L / K}(b_{i}))\).
			Now, we have reduced our question
			to the following claim:
			\(K^{\times} \cap \overline{N_{L / K}(L^{\times})}
			= N_{L / K}(L^{\times}\),
			and this is an exercise in 
			point set topology combined with the 
			strucutre of the units of local fields.

		\item Exercise
		\item Use the characterizing diagram
			and the properties of the cup product, 
			left as an exercise.
	\end{enumerate}
\end{proof}


