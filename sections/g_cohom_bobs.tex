
\section{Group Cohomology Bits and Bobs}

Here, we collect a number of facts about group cohomology
which we will use in our proofs. 

These concepts are used often, and without spending much
time at all on them, so they are included here for 
reference.
For the reader who is new to class field theory, 
spending some time understanding each of these
arguments in-depth might be helpful to following along later.

\subsection{Some Lemmas}

\subsubsection{Shapiro's Lemma}

\subsubsection{Limits and Cohomology}

We will state the following lemma without proof, it has been used before
and it appears in Neukirch as Proposition 1.5.1

\begin{lemma}
	% fix colim
	If \(G = \lim_{i \in I} G_{i}\) and \(A = \colim_{i \in I} A_{i}\),
	then we have 
	\[
		H^{n}(G,A) = \colim_{i \in I} H^{n}(G_{i},A_{i})
	\] 
	for all \(n\)
\end{lemma}

By taking one either the \(G_{i}\) or the \(A_{i}\) to be identically
\(G\) (resp. A), we get the following corollary:

\begin{corollary}\label{cor:lim:cohom}
	\begin{enumerate}[(a)]
		\item If \(G = \lim G_{i}\), then 
			\(H^{n}(G,A) = \colim H^{n}(G_{i},A)\) for a fixed \(A\).
		\item IF \(A = \colim A_{i}\), then 
			\(H^{n}(G,A) = \colim H^{n}(G,A_{i})\) for a fixed \(G\).
	\end{enumerate}
\end{corollary}

It is this corollary that will be most useful to us later.

\subsection{Functorial Properties of Cohomology}
Let $M$ be a $G$-module and $M'$ a $G'$-module. Two maps $\alpha:G'\to G$ and $\beta:M\to M'$ are \textbf{compatible} if $\beta(\alpha(g)m)=g(\beta(m))$. If $(\alpha,\beta)$ is a compatible pair, then $(\alpha,\beta)$ defines a homomorphism of complexes
\begin{align*}
	C^{r}(G,M)&\to C^{r}(G',M')\\
	\phi&\mapsto\beta\circ\phi\circ\alpha^{r}
.\end{align*}
One can check this by chasing the following diagram:
\[
\begin{tikzcd}
	\cdots\ar{r}{d^{r-1}} & C^{r}(G,M)\ar{r}{d^{r}}\ar[d] & C^{r+1}(G,M)\ar{r}{d^{r+1}}\ar[d] & \cdots\\
	\cdots\ar{r}{d^{r-1}} & C^{r}(G',M')\ar{r}{d^{r}} & C^{r+1}(G',M')\ar{r}{d^{r+1}} & \cdots\\
\end{tikzcd}
.\]
The homomorphism of complexes induces a homomorphism on cohomology
\[
	H^{r}(G,M)\to H^{r}(G',M')
.\]
\begin{example}
Let $H$ be a subgroup of $G$. For any $H$-module $M$, the map
\begin{align*}
	\Ind^{G}_H(M)&\to M\\
	\phi&\mapsto \phi(1_G)
\end{align*}
is compatible with the inclusion map $H\hookrightarrow G$. The induced homomorphism
\[
	H^{r}(G,\Ind^{G}_H(M))\to H^{r}(H,M)
\] 
is exactly the isomorphism from Shapiro's lemma.
\end{example}

\subsubsection{Restriction} 

Let $H$ be a subgroup of $G$. Let $\alpha:H\hookrightarrow G$ be the inclusion map and $\beta:M\to M$ be the identity map. Fix $g\in G$, $m\in M$. Since $\beta(\alpha(h)m)=hm=h\beta(m)$, $\alpha$ and $\beta$ are compatible. The induced maps on cohomology are called \textbf{restriction homomorphisms}:
\[
	\res:H^{r}(G,M)\to H^{r}(H,M)
.\] 
Alternatively, $\res$ can be realized as the composition
\[
	H^{r}(G,M)\to H^{r}(G,\Ind^{G}_H(M))\xrightarrow{\overset{\text{Shapiro}}{\simeq}} H^{r}(H,M)
\] 
where the first map is induced by  
\begin{align*}
	M&\to\Ind^{G}_H(M)\\
	m&\mapsto (g\mapsto gm)
\end{align*}
and the second is the isomorphism from Shapiro's lemma.
\subsubsection{Inflation}
Let $H$ be a normal subgroup of $G$. Let $\alpha:G\to G/H$ be the quotient map, and $\beta:M^{H}\hookrightarrow M$ the inlcusion map. Fix $g\in G$, $m\in M^{H}$. Then 
\begin{align*}
	\beta(\alpha(g)m)&=\beta((gH)m)\\
			 &=\beta(g(Hm))\\
			 &=\beta(gm)\\
			 &=gm\\
			 &=g\beta(m)
.\end{align*}
The induced maps on cohomology in this case are called \textbf{inflation homomorphisms}:
\[
	\inf:H^{r}(G/H,M^{H})\to H^{r}(G,M)
.\] 
\subsubsection{Corestriction}
Let $H$ be a subgroup of $G$ of finite index $n$. Let $\left\lbrace s_1,\dots, s_n\right\rbrace$ be a set of left coset representatives for $H$ in $G$ so that $G$ decomposes as
\[
	G=\bigcup\limits_{i=1}^{n}s_iH
.\] 
Let $M$ be a $G$-module. Consider the norm map  
\begin{align*}
	\Nm_{G/H}:M^{H}&\to M\\
	m&\mapsto\sum\limits_{i=1}^{n}s_im
.\end{align*}
Suppose $\left\lbrace s_1',\dots, s_n'\right\rbrace$ is another set of coset representatives for $H$ in $G$. For $m\in M^{H}$,
\begin{align*}
	\sum\limits_{i=1}^{n}s_im&=\sum\limits_{i=1}^{n}s_i's_i'^{-1}(s_im)\\
				 &=\sum\limits_{i=1}^{n}s_i'(s_i'^{-1}s_i)m\\
				 &=\sum\limits_{i=1}^{n}s_i'm
,\end{align*}
so $\Nm_{G/H}$ is independent of choice of coset representatives. Furthermore, for $g\in G$, $\left\lbrace gs_1,\dots, gs_n\right\rbrace$ is again a set of coset representatives. Therefore the norm map is in fact a well-defined homomorphism
\begin{align*}
	\Nm_{G/H}:M^{H}\to M^{G}
.\end{align*}
This can be extended ($M^{\bullet}=H^{0}(\bullet,M)$) to \textbf{corestriction homomorphisms}
\[
	\cor:H^{r}(H,M)\to H^{r}(G,M)
\] 
for all $r$ in the following way: for every $G$-module $M$, there is a canonical $G$-module homomorphism
\begin{align*}
	\Ind^{G}_H(M)&\to M\\
	\phi&\mapsto\sum\limits_{i=1}^{n}s_i\phi(s_i^{-1})
.\end{align*}
The induced map on cohomology, when composed with the isomorphism from Shapiro's lemma, gives $\cor$:
\[
	H^{r}(H,M)\xrightarrow{\simeq}H^{r}(G,\Ind^{G}_H(M))\to H^{r}(G,M)
.\] 
\subsubsection{Some Auxiliary Results}
\begin{proposition}
	For $H$ a subgroup of $G$ of finite index, the composition
	\[
		\cor\circ\res:H^{r}(G,M)\to H^{r}(G,M)
	\] 
	is the multiplication by $[G:H]$ homomorphism.
\end{proposition}
\begin{proof}
	For each $m\in M$, define
	\begin{align*}
		\phi_m:G&\to M\\
		g&\mapsto gm
	.\end{align*}
	Let $S$ be a set of coset representatives for $H$ in $G$, and define
	\begin{align*}
		\psi:\Ind^{G}_H(M)&\to M\\
		\phi&\mapsto\sum\limits_{s\in S}s\phi(s^{-1})
	.\end{align*}
	By the definitions of $\res$ and $\cor$, the composite
	\begin{align*}
		M\xrightarrow{m\mapsto\phi_m}\Ind^{G}_H(M)\xrightarrow{\psi}M
	\end{align*}
	induces exactly the composite $\cor\circ\res$ on cohomology. The result follows since
	\[
		\sum\limits_{s\in S}s\phi_m(s^{-1})=\sum\limits_{s\in S}ss^{-1}m=\sum\limits_{s\in S}m=[G:H]m
	.\] 
\end{proof}
\begin{corollary}\label{cor:mult-triv}
	If $\#G=m$, then $mH^{r}(G,M)=0$ for all $r>0$.
\end{corollary}
\begin{proof}
	By the proposition, the multiplication by $m=[G:\left\lbrace 1\right\rbrace]$ map factors through $H^{r}(\left\lbrace 1\right\rbrace,M)=0$ like so
	\[
		H^{r}(G,M)\xrightarrow{\res}H^{r}(\left\lbrace 1\right\rbrace,M)\xrightarrow{\cor} H^{r}(G,M)
	.\] 
\end{proof}
Let $A$ be an abelian group, $p$ a prime. The subgroup
\[
	A(p)=\left\lbrace g\in A:p^{k}g=0 \text{ for some }k\in\ZZ\right\rbrace
.\] 
\begin{corollary}
	Let $G$ be a finite group, $G_p$ its $p$-Sylow subgroup. For every $G$-module $M$, the restriction map
	\[
		\res:H^{r}(G,M)\to H^{r}(G_p,M)
	\] 
	is injective on the $p$-primary component of $H^{r}(G,M)$.
\end{corollary}
\begin{proof}
	The composite
	\[
		\cor\circ\res:H^{r}(G,M)\to H^{r}(G,M)
	\] 
	is multiplication by $[G:G_p]$, which is not divisible by $p$. It follows then that $\cor\circ\res$ is injective on the $p$-primary component of $H^{r}(G,M)$, so in particular $\res$ is.
\end{proof}

