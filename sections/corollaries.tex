%\documentclass[a4paper]{article}

%% Language and font encodings
%\usepackage[english]{babel}
%\usepackage[utf8x]{inputenc}
%\usepackage[T1]{fontenc}
%
%%% Sets page size and margins
%\usepackage[a4paper,top=3cm,bottom=2cm,left=3cm,right=3cm,marginparwidth=1.75cm]{geometry}
%
%%% Useful packages
%\usepackage{mathptmx,amsmath,colonequals}
%\usepackage{amssymb,gensymb,mathrsfs}
%\usepackage{amscd,amsthm}
%\usepackage{mathtools}
%\usepackage{graphicx}
%\usepackage[colorinlistoftodos]{todonotes}
%\usepackage[colorlinks=true, allcolors=blue]{hyperref}
%\usepackage[shortlabels]{enumitem}
%\usepackage{extarrows}
%\usepackage{tikz,tikz-cd}

%\newtheorem{thm}{Theorem}[section]
%\newtheorem{lem}[thm]{Lemma}
%\newtheorem{cor}[thm]{Corollary}
%\newtheorem{prop}[thm]{Proposition}
%\newtheorem{conj}[thm]{Conjecture}

%\theoremstyle{definition}
%\newtheorem{defn}[thm]{Definition}
%\newtheorem{ex}[thm]{Example}
%\newtheorem{xca}[thm]{Exercise}

%\theoremstyle{remark}
%\newtheorem{rmk}[thm]{Remark}

\newcommand{\category}[1]{\textsf{#1}}


% commalg specific?
%\newcommand{\Hom}{\operatorname{Hom}}

%\newcommand{\depth}{\operatorname{depth}}
%\newcommand{\height}{\operatorname{height}}
%\newcommand{\pd}{\operatorname{pd}}
%\newcommand{\injd}{\operatorname{injd}}

% algebraic geometry
%\newcommand{\Spec}{\operatorname{Spec}}

% custom ops here
%\newcommand{\Gal}{\operatorname{Gal}}



% custom commands here

\title{}
%\author{Jack J Garzella}

\begin{document}

%\maketitle

\section{Some Cohomological Corollaries}

At this point we have completely the first major 
milestone, calculating the Brauer group.
Now, we deduce many consequences of this theorem 
in terms of group cohomology that we will need later.
%Each of the proofs is more or less straightforward
%applications of the machinery of group cohomology,
%so if the reader is familiar with these methods
%or wants to save the gory details for later, the 
%proofs can be skipped.

First, we consider a finite extension \(L / K\). 
Hilbert Theorem 90 gives 
$H^{1}(G_L,\overline{L}^{\times})=\left\lbrace1\right\rbrace$, 
so we can use dimension shifting or the Hochschild-Serre spectral sequence 
to extend the inflation-restriction exact sequence. 
This combined with the theorem gives
\[
\begin{tikzcd}
	0\ar[r] & H^{2}(G(L/K),\overline{L}^{\times})\ar[r] & H^{2}(G_K,\overline{K}^{\times})\ar{r}{\res}\ar[d,"\inv_K"',"\simeq"] & H^{2}(G_L,\overline{L}^{\times})\ar[d,"\inv_L","\simeq"']\\
		& & \QQ/\ZZ\ar{r}[swap]{[L:K]} & \QQ/\ZZ
\end{tikzcd}
.\] 

Therefore we have
\begin{corollary}[Is this used anywhere?] \coproof
	\[
		H^{2}(\Gal(L/K),\overline{L}^{\times})
		\simeq\frac{1}{[L:K]}\ZZ/\ZZ.
	\]
\end{corollary}

Let us prove a few more corollaries
\begin{corollary} \coproof
$\text{inv}_K: H^2(G_K, \mu_n) \xrightarrow{\sim} 
\frac{1}{n}\mathbb{Z}/\mathbb{Z}$ is an isomorphism, where $\mu_n$ denotes the group of $n$th roots of unity in $K$.
\end{corollary}
\begin{proof}
Consider the exact sequence 
$$1\rightarrow \mu_n\rightarrow 
\bar{K}^\times\xrightarrow{\cdot n}\bar{K}^\times\rightarrow 1$$ 
where the map from 
$\bar{K}^\times$ 
to $\bar{K}^\times$ 
is multiplication by $n$. 
Passing to the exact sequence induced by cohomology, we have 
by Hilbert Theorem 90
$$0\rightarrow H^2(G_K,\mu_n)\xrightarrow{\sim} \text{Br}(K)[n]$$ 
From the above, we know $\text{Br}(K)[N]=\frac{1}{n}\mathbb{Z}/\mathbb{Z}$
\end{proof}

%\begin{corollary}
%\begin{enumerate}
%    \item Br$(K^{ur})=0$
%    \item $H^2(I_K,\mu_n)=0\forall n\geq 0$
%    \item If $M$ is a torsion $I_K$ module, then $H^i(I_K, M)=0\forall i\geq 2$
%    \item if $M$ is a finite $G_K$-module, then $|H^2(I_K,\mu_n)|$ is finite $\forall n\geq 0$ and $H^i(G_K, M)=0\forall i\geq 3$.
%\end{enumerate}
%\end{corollary}
%
%
%We'll start with a few corollaries of the calculation of the Braur Group.
%
%Corollary 1 in Wei's talk

\begin{corollary} \coproof
	\(Br(K^{unr}) = 0\)
\end{corollary}

\begin{proof}
		\item The algebraic closure of \(K^{unr}\) is just \(\overline{K}\),
			so we are trying to compute 
			\(H^{2}(G_{\overline{K} / K^{unr}},\overline{K}^{\times})\).
			First, we note that
			 \[
			 \lim_{K \ins L \ins K^{unr}} G_{\overline{K} / L} 
			 = G_{\overline{K} / K^{unr}}
			 \] 
			where the limit is taken over the restriction maps.
			This extends to a limit of cohomology by 
			Corollary \ref{cor:lim:cohom} part (a), and
			we have
			\[
				\colim_{K \ins L \ins K^{unr}} H^{i}(G_{\overline{K} / L}) 
				= H^{i}(G_{\overline{K} / K^{unr}})
			.\] 
			And we notice that the right hand side we can replace using
			our commutative diagram to get
			\[
				\colim_{[L:L^{\prime}]} \mathbb{Q} / \mathbb{Z}
			\] 
			This limit is zero, and we can see this through the coproduct
			/coequalizer characterization of the colimit. 
			A coequalizer is like taking a quotient, and vaguely, the 
			equivalence relation is those things which eventually end up being
			the same under these multiplication maps. 
			But the maps are multipication by an (arbitrarily large) constant,
			and the module is torsion.
	
\end{proof}

\begin{corollary} \coproof
Let $I_K$ be the inertia group. Then
	\(H^{2}(I_{K},\mu_{n}) = 0\) for all \(n\) greater than 
	or equal to \(0\)
\end{corollary}

\begin{proof} 
		\item Use the exact sequence (of modules) 
			\[
			1 \to \mu_{n} \to \overline{K}^{\cross} \to \overline{K}^{\cross}
			\to 1
			\] 
			Then take the long exact sequence on cohomology, when we are using
			\(I_{K}\) as the group for the cohomology.
			We get 
			\[
				\ldots \to H^{1}(I_{K},\overline{K}^{\cross}) \to
				H^{2}(I_{K}, \mu_{n}) \to H^{2}(I_{K},\overline{K}^{\cross})
				\to \ldots
			\] 
			Now, we need to remember that 
            \(I_{K} \isom \Gal\left( \overline{K} / K^{unr} \right)\).
			With this in mind, we see that
			the group on the left is zero by Hilbert Theorem 90,
			%TODO:{\color{red} This isn't the exact formulation,
			%check precisely how hilb90 imply this}
			%, 
			and
			the first part of the proposition
			says exactly that the group on the right is zero. 
			We conclude our statement by exactness.
\end{proof}

\begin{proposition} \coproof
		 If \(M\) is a torsion \(I_{K}\)-module, then
			\(H^{i}(I_{K},M) = 0\) for all \(i\) with \(2 \leq i\)
\end{proposition}

\begin{proof}
			It suffices to consider
			finite (and thus torsion) modules, by Corollary \ref{cor:lim:cohom}.
			First, we will reduce to cohomological dimension 2 by dimension
			shifting. 
			Assume \(H^{2}(M) = 0\) for all torsion modules \(M\).
			First, fix \(M\) and embed it into an injective module \(I\) 
			which is also torsion.
			%TODO:{\color{red} perhaps justify why we can do this?}
			(Think \(\mathbb{Q} / \mathbb{Z}\)).
			Now, the quotient will be torsion and we have the short exact sequence
			\[
			\begin{tikzcd}
			0 \arrow{r}{} & M \arrow{r}{} & 
			I \arrow{r}{} & I / M \arrow{r}{} & 0
			\end{tikzcd}
			\]
			Now, taking long exact sequences, we get
			\[
				\ldots \to H^{2}(\frac{I}{M}) \to 
				H^{3}(M) \to H^{3}(I) \to \ldots
			\] 
			and by assumption, the \(H^{2}(I / M)\) must be zero.
			As \(I\) is injective, its \(H^{3}\) is zero and we conclude
			that \(H^{3}(M)\) is zero. 
			By induction, extend this to \(H^{i}(M)\) for \(2 \leq i\).

			Finally, we will show that \(H^{2}(I_{K},M)\) is zero 
			for a finite torsion module \(M\) :
			Without loss of generality, (by Chinese Remainer Theorem)
			assume that \(M\) is \(l^{\infty}\)-torsion for some prime 
			\(l\).
			Then, by {\color{red} reference?} Sriram's corollary 1.2 and taking limits by
			Corollary \ref{cor:lim:cohom}, we have that the restriction
			map \(res : H^{2}(I_{K},M) \to H^{2}(\Syl_{l}(I_{K}),M)\),
			where \(\Syl_{l}(I_{K})\) is (any) pro-\(l\) Sylow Subgroup,
			is injective. 
			Thus it suffices to show for \(\Syl_{l}\left( I_{K} \right)\).
			By induction on the size of \(M\), we reduce to the case of
			\(M = \mu_{l}\).
			(every minimal, thus indecomposable, module of 
			\(\Syl_{l}(I_{K})\) must be isomorphic to
			\(\mathbb{Z} / l\mathbb{Z} \isom \mu_{l}\))

			But now, we notice that \(\Syl_{l}(I_{K})\) is the intersection
			of the open subgroups containing it (the colimit). 
			This will give us that
			\[
				H^{2}(\Syl_{l}(I_{K}),\mu_{l}) = 
				\colim H^{2}\left( I_{L},\mu_{l} \right)
			\] 
			where the colimit is over finite extensions \(L / K\) where
			\(l \nmid e_{L / K}\).
			As the \(l\) does not divide the ramification index, every
			element of \(I_{K}\) has order coprime to \(l\), giving us that 
			the group is zero.
\end{proof}


\begin{proposition} \label{prop:finite:module}
%Notation as above, let $I_K$ be the inertia group, then
If \(M\) is a finite \(G_{K}\)-module, then
\begin{enumerate}[(i)]
	\item For all \(i\), \(\left| H^{i}(G_{K},M) \right| < \infty\) 
	\item \coproof For \(i\) with \(3 \leq i\), 
			\(H^{i}(G_{K},M) = 0\).
\end{enumerate}
\end{proposition}



\begin{proof}
			We will use spectral sequences to calculate the cohomology
			of \(G_{K}\). 
			Our spectral sequence in this context will always "come from"
			a "composition of operations".
			%Precisely this means a composition of \textit{derived functors},
			%or functors in the \textit{derived category}, but we do
			%not want to bother developing this (quite technical)
			%machinery, so we will limit ourselves to the following situation.

			When we have an exact sequence of groups
			\[
			1 \to H \to G_{K} \to G_{K} / H \to 1	
			,\] 
			then we can notice that if we take the invariants of \(H\), and
			then take the invariants of \(G_{K} / H\), we will get
			the invariants of \(G\), by straightforward application of
			the definitions of invariants.

			This ``composition'', i.e. first taking one invariants and 
			then the other, will give us a spectral sequence.
			We will use two such exact sequences:
			\begin{enumerate}[(1)]
				\item \(1 \to I_{K} \to G_{K} \to \Gal(K^{unr} / K) \to 1\) 
				\item \(1 \to G_{L} \to G_{K} \to \Gal\left( L / K \right) \to 1\) 
					for a finite \(L / K\).
			\end{enumerate}
					
			First, we use the spectral sequence we get from (1), which
			gives 
			\[
				H^{i}\left( \Gal(K^{unr} / K), 
				H^{j}\left( I_{K}, M \right)\right) 
				\implies H^{i + j}\left( G_{K},M \right)
			.\] 
			By our previous statement, the inside term vanishes when
			\(2 \leq j\), and will always be torsion.
			Remember that \(M\) is finite, and thus torsion.
			For \(2 \leq i\), we can take our coefficients mod torsion,
			so these will also be zero. % check why
			% I think the K^unr / K has to have index prime
			% to the torsion
			We now have in our spectral sequence grid picture that 
			each of the grid points is zero except for 
			(0,0),(0,1),(1,0), and (1,1).
			It is then straightforward to see that \(H^{3}\) and above
			will vanish.

			Now, for finiteness, consider the other sequence
			\[
				H^{i}\left( \Gal\left( L / K \right),
				H^{j}\left( G_{L},M \right)\right)
				\implies H^{i+j}\left( G_{K},M \right)
			.\] 
			We have that \(\Gal\left( L / K \right)\) is clearly finite,
			and we will show that \(H^{j}\left( G_{L},M \right)\) is
			finite for a specific choice of \(L\).
			Then, all of the grid points on the spectral sequence will
			be finite, and thus termination in finite steps will give 
			finiteness.

			So, choose \(L\) such that the action of \(G_{L}\) on \(M\) 
			is trivial, as is that of \(G_{L}\) on \(\mu_{\left| M \right|}\).
			By our choice of \(L\), \(M\) is isomorphic (as a
			\(G_{L}\)-module) to a direct sum of \(\mu_{n_{i}}\) for 
			some \(n_{i}\)'s.
			By the fact that direct sums are colimits, we can use
			our favorite Corollary \ref{cor:lim:cohom} to split
			the computation up for a single \(\mu_{n}\).

			For \(j=0\), the module is clearly finite (we are taking
			invariants of a finite module).
			For \(j=1\), we have Kummer Theory
			telling us that \(H^{1}\) is isomorphic
			to the finite group \(K^{\times} / (K^{\times})^{n}\).
			For \(j = 2\), we have the exact sequence
			\[
			1 \to \mu_{n} \to \overline{L}^{\cross} \to 
			\overline{L}^{\cross} \to 1
			\] 
			and the computation of the Brauer group.
			For \(3 \leq j\), the cohomology group vanishes, 
			as above.

\end{proof}

\end{document}
