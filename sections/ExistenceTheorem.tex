%\documentclass[11pt]{article}
%\usepackage[utf8]{inputenc}
%\usepackage{amsmath,amsfonts,amssymb,amsthm,tikz-cd,mathpazo,euler,mathtools}
%\usetikzlibrary{arrows}
%\tikzset{commutative diagrams/.cd,arrow style=tikz,diagrams={>=latex'}}
%\usepackage[shortlabels]{enumitem}
%\setlength\parindent{0pt}

%\newcommand{\ZZ}{\mathbb{Z}}
%\newcommand{\CC}{\mathbb{C}}
%\newcommand{\QQ}{\mathbb{Q}}
%\newcommand{\FF}{\mathbb{F}}
%\newcommand{\NN}{\mathbb{N}}
%\newcommand{\SL}{\operatorname{SL}}
%\newcommand{\Div}{\operatorname{Div}}
%\newcommand{\Hom}{\operatorname{Hom}}
%\newcommand{\Spec}{\operatorname{Spec}}
%\newcommand{\Fun}{\operatorname{Fun}}
%\newcommand{\Nat}{\operatorname{Nat}}
%\newcommand{\Aut}{\operatorname{Aut}}
%\newcommand{\Gal}{\operatorname{Gal}}
%\newcommand{\Frob}{\operatorname{Frob}}
%\newcommand{\Ind}{\operatorname{Ind}}
%\newcommand{\res}{\operatorname{res}}
%\newcommand{\cor}{\operatorname{cor}}
%\newcommand{\inv}{\operatorname{inv}}
%\newcommand{\ord}{\operatorname{ord}}
%\newcommand{\ev}{\operatorname{ev}}
%\newcommand{\GL}{\operatorname{GL}}
%\newcommand{\Br}{\operatorname{Br}}
%\newcommand{\C}{\operatorname{C}}
%\newcommand{\B}{\mathrm{B}}
%\newcommand{\Nm}{\mathrm{Nm}}
%\newcommand{\im}{\mathrm{im}}
%\newcommand{\frakm}{\mathfrak{m}}
%\newcommand{\frakn}{\mathfrak{n}}
%\newcommand{\fraka}{\mathfrak{a}}
%\newcommand{\frakb}{\mathfrak{b}}
%\newcommand{\frakc}{\mathfrak{c}}
%\newcommand{\frakp}{\mathfrak{p}}
%\newcommand{\frakq}{\mathfrak{q}}
%\newcommand{\frakP}{\mathfrak{P}}
%\newcommand{\frakO}{\mathfrak{O}}
%
%\theoremstyle{definition}
%\newtheorem{definition}{Definition}[section]
%\newtheorem{lemma}{Lemma}[section]
%\newtheorem{theorem}{Theorem}[section]
%\newtheorem{proposition}{Proposition}[section]
%\newtheorem{corollary}{Corollary}[section]
%\newtheorem{remark}{Remark}[section]
%\newtheorem{example}{Example}[section]
%\newtheorem{conjecture}{Conjecture}[section]

\begin{document}

\section{The Local Existence Theorem}


\begin{theorem}
	For a local field $K$, there is a homomorphism $r_K:K^{\times}\to G_K^{ab}$ with dense image characterized by the commutativity of the diagram
	\[
	\begin{tikzcd}
		& \Hom(K^{\times},\ZZ/n\ZZ)\times K^{\times}/(K^{\times})^{n}\ar[d,shift left=3.5em,"\text{Kummer}","\simeq"']\ar{r}{\text{eval}} & \ZZ/n\ZZ\ar[hookrightarrow]{d}{\cdot\frac{1}{n}}\\
		\Hom(G_k^{ab},\ZZ/n\ZZ)\ar{ur}{r_K^{*}}\ar[equals]{r} & H^{1}(K,\ZZ/n\ZZ)\times H^{1}(K,\mu_n)\ar{r}{\langle\ ,\ \rangle_K} & \QQ/\ZZ 
	\end{tikzcd}
	.\] 
	It satisfies
	\begin{enumerate}[(1)]
		\setcounter{enumi}{4}
		\item The diagram
			\[
			\begin{tikzcd}
				1\ar[r]&\mathcal{O}_K^{\times}\ar[r]\ar[d,"\simeq"]&K^{\times}\ar[r]\ar[d,hookrightarrow,"r_k"]&\ZZ\ar[r]\ar[d,hookrightarrow]&0\\
				1\ar[r]&I_{K^{ab}}\ar[r]&G_K^{ab}\ar[r]&G_k^{ab}\ar[r]&0
			\end{tikzcd}
			.\] 
		\item (Existence Theorem) The open subgroups of finite index in $K^{\times}$ are precisely the norm groups. Moreover, $L\mapsto N_{L/K}(L^{\times})$ gives a one-to-one correspondence between finite abelian extensions of $K$ and open subgroups of finite index satisfying
			\begin{itemize}
				\item $L_1\subseteq L_2$ if and only if $N(L_2^{\times})\subseteq N(L_1^{\times})$,
				\item $N_{L_1L_2}=N_{L_1}\cap N_{L_2}$, and
				\item $N_{L_1\cap L_2}=N_{L_1}\cdot N_{L_2}$.
			\end{itemize}
	\end{enumerate}
\end{theorem}
\begin{proof}
	We first prove (5). We first check commutativity of the diagram
	\[
	\begin{tikzcd}
		K^{\times}\ar[r,"\nu_k"]\ar[d] & \ZZ\ar[r]\ar[d] & 0\\
		G_k^{ab}\ar[r] & G_k\simeq G(K^{ur}/K)\ar[r,"\simeq"] & 1
	\end{tikzcd}
	.\] 
	Let $a\in K^{\times}$, $\chi\in H^{1}(G(K^{ur}/K),\QQ/\ZZ)$. We need to show that
	\[
		\chi(r_k(a))=\chi\left(\Frob_k^{-\nu_k(a)}\right)
	.\] 
	We know that $\chi(r_k(a))=-\inv_K(a\smile d\chi)$. The diagram
	\[
		\begin{tikzcd}[column sep = 1em]
		H^{0}(G(K^{ur}/K),(K^{ur})^{\times})\times H^{2}(G(K^{ur}/K),\ZZ)\ar[r,"\smile"] & H^{2}(G(K^{ur}/K),(K^{ur})^{\times})\ar[r,"\inf","\simeq"']\ar[d,"\nu_K"] & \Br(K)\ar[d,"\inv_K"]\\
												 & H^{2}(G(K^{ur}/K),\ZZ) & \QQ/\ZZ\\
												 & H^{1}(G(K^{ur}/K),\QQ/\ZZ)\ar[ru,"\ev"']\ar[u,"d"',"\simeq"] & 
	\end{tikzcd}
	\] 
	commutes, so we see that 
	\begin{align*}
		\chi(r_K(a))&=-\inv_K(a\smile d\chi)\\
			    &=-\ev\circ d^{-1}\circ\nu_K(a\smile d\chi)\\
			    &=-\ev(\nu_K(a)\cdot\chi)\\
			    &=\chi\left(\Frob_k^{-\nu_K(a)}\right)
	.\end{align*}
	This diagram then induces
	\[
	\begin{tikzcd}
		1\ar[r]&\mathcal{O}_K^{\times}\ar[r]\ar[d]&K^{\times}\ar[r]\ar[d,hookrightarrow,"r_k"]&\ZZ\ar[r]\ar[d,hookrightarrow]&0\\
		1\ar[r]&G(K^{ab}/K^{ur})\ar[d,equals]\ar[r]&G_K^{ab}\ar[r]&G_k^{ab}\ar[r]&0\\
		       & I_{K^{ab}} & & &
	\end{tikzcd}
	.\] 
	We claim that $r_K\mid_{\mathcal{O}_K^{\times}}$ is an isomorphism. $r_k$ is injective, so we need to prove surjectivity. Let $L/K$ be a finite abelian extension. Consider the tower
	\[
	\begin{tikzcd}
		L\ar[d,-,"I_{L/K}"]\\
		M\ar[d,-,"\text{unr.}"]\\
		K
	\end{tikzcd}
	.\] 
	Fix $h\in I_{L/K}$. We know that there exists $a\in K^{\times}$ such that $r_K(a)=h$. By (5),
	\[
		\Frob_K^{-\nu_K(a)}\mid_M=r_K(a)\mid_M=h(a)\mid_M=1
	.\] 
	It follows that $f_{L/K}\mid\nu_K(a)$. Thus, there exists some $b\in L^{\times}$ such that $\nu_K(a)=f_{L/K}\cdot\nu_L(b)=\nu_K(N(b))$. But then $u=a\cdot N(b)^{-1}\in\mathcal{O}_K^{\times}$, and $r_K(u)=h$. $r_K(\mathcal{O}_K^{\times})$ is then dense in $I_{K^{ab}}$. Since $\mathcal{O}_K^{\times}$ is compact and $r_K$ is continuous, $r_K(\mathcal{O}_K^{\times})$ is also compact. In particular, it is closed, so $r_K(\mathcal{O}_K^{\times})=I_K^{ab}$.\\
	We now prove (6). We have seen that $\mathcal{O}_K^{\times}\xrightarrow{r_K}I_{K^{ab}}$ is an isomorphism. In fact since $\mathcal{O}_K^{\times}$  is compact, $I_{K^{ab}}$ is Hausdorff, and $r_K$ is a continuous bijection, $r_K$ must be a homeomorphism.\\
	For a finite abelian extension $L/K$, we then have
	\[
		\mathcal{O}_K^{\times}\xrightarrow{\text{homeo.}}I_{K^{ab}}=\varprojlim_{\stackrel{L/K}{\text{fin., ab.}}}I_{L/K}\cong\varprojlim_{\stackrel{L/K}{\text{fin., ab.}}}\mathcal{O}_K^{\times}/\left(N(L^{\times})\cap\mathcal{O}_K^{\times}\right)
	.\] 
	So the topology on $\mathcal{O}_K^{\times}$ induced by norm subgroups is the usual profinite topology. That is, for a (profinite) open $U\subseteq\mathcal{O}_K^{\times}$, the map $\mathcal{O}_K^{\times}\twoheadrightarrow\mathcal{O}_K^{\times}/U$ factors
	\[
	\begin{tikzcd}
		\mathcal{O}_K^{\times}\ar[r,two heads]\ar[dr,dashed] & \mathcal{O}_K^{\times}/U\\
								   & \mathcal{O}_K^{\times}/\left(N(L^{\times})\cap\mathcal{O}_K^{\times}\right)\ar[u,dashed]
	\end{tikzcd}
	\] 
	for some $L/K$ finite and abelian. Now let $U\subset K^{\times}$ be open and of finite index. $U\cap\mathcal{O}_K^{\times}$ is then open and of finite index and
	\[
		\nu_K(U)\supset m\ZZ\text{ for some } m\ge 1
	.\] 
	There exists some finite abelian extension $L/K$ such that 
	\[
		N(L^{\times})\cap\mathcal{O}_K^{\times}\subset U\cap\mathcal{O}_K^{\times}
	.\] 
	By replacing $L$ by the composite of $L$ and an unramified extension of sufficiently high degree, we may assume that $\nu_K(N(L^{\times}))\subset\nu_K(U)$. Then
	\[
		N(L^{\times})\subset U\subset K^{\times}
	\] 
	and everything is open and of finite index. Now we need to show that $U$ is itself a norm group. We have a commutative diagram
	\[
	\begin{tikzcd}
		K^{\times}/N(L^{\times})\ar[r,"\simeq","r_{L/K}"']\ar[d,two heads] & G(L/K)\ar[d, two heads, "\text{Gal. Theory}"]\\
		K^{\times}/U\ar[r,"\simeq","r_K"'] & G(M/K) & K^{\times}/N(M^{\times})\ar[l,"r_K","\simeq"']
	\end{tikzcd}
	\] 
	for some tower
	\[
	\begin{tikzcd}
		L\ar[d,-]\\
		M\ar[d,-,"\text{Galois}"]\\
		K
	\end{tikzcd}
	\] 
	This shows that $U=N_{M/K}(M^{\times})$, as desired.
\end{proof}
\end{document}

